\section{Answers Lecture 1: Probability, Vectors, Differentiation}

\paragraph{\questionref{q:vecnot}} 

Given $p(\textbf{x}) = \frac{1}{C}(x_1^2 + x_1x_2 x_2^2 + 2x_2x_3)$, we need to rearrange the terms to find an expression as follows:
\begin{equation}
p(\textbf{x}) = \frac{1}{C}(\textbf{x}^T A \textbf{x}), \quad A\in \mathbb{R}^{3\times 3}.
\end{equation}
Inspection of the terms in $p(\textbf{x})$ gives the following solution
\begin{align*}
\begin{pmatrix}
x_1 & x_2 & x_3
\end{pmatrix}^T
\begin{pmatrix}
1 & \frac{1}{2} & 0\\
\frac{1}{2} & 1 & 1\\
0 & 1 & 0
\end{pmatrix}
\begin{pmatrix}
x_1 & x_2 & x_3
\end{pmatrix} =
\begin{pmatrix}
x_1 + \frac{x_2}{2}\\
\frac{x_1}{2} + x_2 + x_3\\
x_2
\end{pmatrix}
\begin{pmatrix}
x_1 & x_2 & x_3
\end{pmatrix} = x_1^2 + x_1x_2 x_2^2 + 2x_2x_3
.\end{align*}
Thus,
\begin{equation}
p(\textbf{x}) = \frac{1}{C}(\textbf{x}^T A \textbf{x}), \quad A = \begin{pmatrix}
1 & \frac{1}{2} & 0\\
\frac{1}{2} & 1 & 1\\
0 & 1 & 0
\end{pmatrix}.
\end{equation}


\paragraph{\questionref{q:noisecondind}}

Given inputs $X\in\mathbb{R}^{D\times N}$, where $X=\{\textbf{x}_1, \dots, \textbf{x}_N\}$, we want to show that all $y_n$s are independent, i.e,
\begin{align}
p(\vy|\vtheta,\mat X) = \prod_{n=1}^N p(y_n|\vtheta,\textbf{x}_n).
\end{align}

We can show this by rearranging terms of the Gaussian distribution:
\begin{align}
p(\vy|\vtheta,\mat X) &= \NormDist{\vy; \vtheta\transpose X, \sigma^2 \eye} = \frac{1}{\sqrt{(2\pi)^N|\sigma^2 I|}}\exp\big(-\frac{1}{2}(\textbf{y} - \vtheta\transpose X)\transpose\sigma^{-2}I (\textbf{y} - \vtheta\transpose X)\big) \\
&= \frac{1}{\sqrt{(2\pi\sigma^2)^N}}\exp\big(-\frac{1}{2\sigma^2}||\textbf{y} - \vtheta\transpose X||^2\big) = \frac{1}{\sqrt{(2\pi\sigma^2)^N}}\exp\big(-\frac{1}{2\sigma^2}\sum_{n=1}^N(y_n - \vtheta\transpose \textbf{x}_n)^2\big) \\
&= \prod_{n=1}^N \frac{1}{\sqrt{2\pi\sigma^2}}\exp\big(-\frac{1}{2\sigma^2}(y_n - \vtheta\transpose \textbf{x}_n)^2\big) = \prod_{n=1}^N p(y_n|\vtheta,\textbf{x}_n)
.\end{align}
