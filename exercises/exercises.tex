\documentclass[a4paper]{article}




%% Language and font encodings
\usepackage[english]{babel}
\usepackage[utf8]{inputenc}
\usepackage[T1]{fontenc}

%% Sets page size and margins
\usepackage[a4paper,top=2.5cm,bottom=2.5cm,left=2.5cm,right=2.5cm,marginparwidth=1.75cm]{geometry}

%% Useful packages
\usepackage{multicol}
\usepackage{paralist}
\usepackage{amsmath}
\usepackage{bm,bbm}
\usepackage{amsthm}
\usepackage{graphicx}
\usepackage[colorinlistoftodos]{todonotes}
\usepackage[colorlinks=true, allcolors=blue]{hyperref}
\usepackage{nameref,cleveref}

\usepackage{../includes/MarkMathCmds}
\usepackage{mathtools}
\newcommand{\mat}[1]{{\mathrm{{#1}}}} % matrix

\usepackage{MarkBiblatexCmds}
\addbibresource{ref.bib}

\usepackage{enumitem}
\usepackage{epsdice}

% tikzlibrary.code.tex
%
% Copyright 2010-2011 by Laura Dietz
% Copyright 2012 by Jaakko Luttinen
%
% The MIT License
%
% See LICENSE file for more details.

% Load other libraries
\usetikzlibrary{shapes}
\usetikzlibrary{fit}
\usetikzlibrary{chains}
\usetikzlibrary{arrows}

% Latent node
\tikzstyle{latent} = [circle,fill=white,draw=black,inner sep=1pt,
minimum size=20pt, font=\fontsize{10}{10}\selectfont, node distance=1]
% Observed node
\tikzstyle{obs} = [latent,fill=gray!25]
% Constant node
\tikzstyle{const} = [rectangle, inner sep=0pt, node distance=1]
% Factor node
\tikzstyle{factor} = [rectangle, fill=black,minimum size=5pt, inner
sep=0pt, node distance=0.4]
% Deterministic node
\tikzstyle{det} = [latent, diamond]

% Plate node
\tikzstyle{plate} = [draw, rectangle, rounded corners, fit=#1]
% Invisible wrapper node
\tikzstyle{wrap} = [inner sep=0pt, fit=#1]
% Gate
\tikzstyle{gate} = [draw, rectangle, dashed, fit=#1]

% Caption node
\tikzstyle{caption} = [font=\footnotesize, node distance=0] %
\tikzstyle{plate caption} = [caption, node distance=0, inner sep=0pt,
below left=5pt and 0pt of #1.south east] %
\tikzstyle{factor caption} = [caption] %
\tikzstyle{every label} += [caption] %

%\pgfdeclarelayer{b}
%\pgfdeclarelayer{f}
%\pgfsetlayers{b,main,f}

% \factoredge [options] {inputs} {factors} {outputs}
\newcommand{\factoredge}[4][]{ %
  % Connect all nodes #2 to all nodes #4 via all factors #3.
  \foreach \f in {#3} { %
    \foreach \x in {#2} { %
      \path (\x) edge[-,#1] (\f) ; %
      %\draw[-,#1] (\x) edge[-] (\f) ; %
    } ;
    \foreach \y in {#4} { %
      \path (\f) edge[->, >={triangle 45}, #1] (\y) ; %
      %\draw[->,#1] (\f) -- (\y) ; %
    } ;
  } ;
}

% \edge [options] {inputs} {outputs}
\newcommand{\edge}[3][]{ %
  % Connect all nodes #2 to all nodes #3.
  \foreach \x in {#2} { %
    \foreach \y in {#3} { %
      \path (\x) edge [->, >={triangle 45}, #1] (\y) ;%
      %\draw[->,#1] (\x) -- (\y) ;%
    } ;
  } ;
}

% \factor [options] {name} {caption} {inputs} {outputs}
\newcommand{\factor}[5][]{ %
  % Draw the factor node. Use alias to allow empty names.
  \node[factor, label={[name=#2-caption]#3}, name=#2, #1,
  alias=#2-alias] {} ; %
  % Connect all inputs to outputs via this factor
  \factoredge {#4} {#2-alias} {#5} ; %
}

% \plate [options] {name} {fitlist} {caption}
\newcommand{\plate}[4][]{ %
  \node[wrap=#3] (#2-wrap) {}; %
  \node[plate caption=#2-wrap] (#2-caption) {#4}; %
  \node[plate=(#2-wrap)(#2-caption), #1] (#2) {}; %
}

% \gate [options] {name} {fitlist} {inputs}
\newcommand{\gate}[4][]{ %
  \node[gate=#3, name=#2, #1, alias=#2-alias] {}; %
  \foreach \x in {#4} { %
    \draw [-*,thick] (\x) -- (#2-alias); %
  } ;%
}

% \vgate {name} {fitlist-left} {caption-left} {fitlist-right}
% {caption-right} {inputs}
\newcommand{\vgate}[6]{ %
  % Wrap the left and right parts
  \node[wrap=#2] (#1-left) {}; %
  \node[wrap=#4] (#1-right) {}; %
  % Draw the gate
  \node[gate=(#1-left)(#1-right)] (#1) {}; %
  % Add captions
  \node[caption, below left=of #1.north ] (#1-left-caption)
  {#3}; %
  \node[caption, below right=of #1.north ] (#1-right-caption)
  {#5}; %
  % Draw middle separation
  \draw [-, dashed] (#1.north) -- (#1.south); %
  % Draw inputs
  \foreach \x in {#6} { %
    \draw [-*,thick] (\x) -- (#1); %
  } ;%
}

% \hgate {name} {fitlist-top} {caption-top} {fitlist-bottom}
% {caption-bottom} {inputs}
\newcommand{\hgate}[6]{ %
  % Wrap the left and right parts
  \node[wrap=#2] (#1-top) {}; %
  \node[wrap=#4] (#1-bottom) {}; %
  % Draw the gate
  \node[gate=(#1-top)(#1-bottom)] (#1) {}; %
  % Add captions
  \node[caption, above right=of #1.west ] (#1-top-caption)
  {#3}; %
  \node[caption, below right=of #1.west ] (#1-bottom-caption)
  {#5}; %
  % Draw middle separation
  \draw [-, dashed] (#1.west) -- (#1.east); %
  % Draw inputs
  \foreach \x in {#6} { %
    \draw [-*,thick] (\x) -- (#1); %
  } ;%
}


% Copyright (C) 2016  Joseph Rabinoff

% ipe2tikz is free software; you can redistribute it and/or modify it under
% the terms of the GNU General Public License as published by the Free
% Software Foundation; either version 3 of the License, or (at your option)
% any later version.

% ipe2tikz is distributed in the hope that it will be useful, but WITHOUT ANY
% WARRANTY; without even the implied warranty of MERCHANTABILITY or FITNESS
% FOR A PARTICULAR PURPOSE.  See the GNU General Public License for more
% details.

% You should have received a copy of the GNU General Public License along with
% ipe2tikz; if not, you can find it at "http://www.gnu.org/copyleft/gpl.html",
% or write to the Free Software Foundation, Inc., 675 Mass Ave, Cambridge, MA
% 02139, USA.


% ipe compatibility TikZ styles

\usetikzlibrary{arrows.meta}

\makeatletter

% These should behave almost exactly like ipe arrows.  They disable correcting
% for the miter length and line width.  This is important for visual consistency
% with ipe, since ipe arrows get much larger when the line width is increased.
% They also use the line join and cap styles from the main path.  These are very
% simple arrows: there is no harpoon version, and the convex hull computation is
% sloppy.

\pgfdeclarearrow{
  name = ipe _linear,
  defaults = {
    length = +1bp,
    width  = +.666bp,
    line width = +0pt 1,
  },
  setup code = {
    % Control points
    \pgfarrowssetbackend{0pt}
    \pgfarrowssetvisualbackend{
      \pgfarrowlength\advance\pgf@x by-.5\pgfarrowlinewidth}
    \pgfarrowssetlineend{\pgfarrowlength}
    \ifpgfarrowreversed
      \pgfarrowssetlineend{\pgfarrowlength\advance\pgf@x by-.5\pgfarrowlinewidth}
    \fi
    \pgfarrowssettipend{\pgfarrowlength}
    % Convex hull
    \pgfarrowshullpoint{\pgfarrowlength}{0pt}
    \pgfarrowsupperhullpoint{0pt}{.5\pgfarrowwidth}
    % The following are needed in the code:
    \pgfarrowssavethe\pgfarrowlinewidth
    \pgfarrowssavethe\pgfarrowlength
    \pgfarrowssavethe\pgfarrowwidth
  },
  drawing code = {
    \pgfsetdash{}{+0pt}
    \ifdim\pgfarrowlinewidth=\pgflinewidth\else\pgfsetlinewidth{+\pgfarrowlinewidth}\fi
    \pgfpathmoveto{\pgfqpoint{0pt}{.5\pgfarrowwidth}}
    \pgfpathlineto{\pgfqpoint{\pgfarrowlength}{0pt}}
    \pgfpathlineto{\pgfqpoint{0pt}{-.5\pgfarrowwidth}}
    \pgfusepathqstroke
  },
  parameters = {
    \the\pgfarrowlinewidth,%
    \the\pgfarrowlength,%
    \the\pgfarrowwidth,%
  },
}


\pgfdeclarearrow{
  name = ipe _pointed,
  defaults = {
    length = +1bp,
    width  = +.666bp,
    inset  = +.2bp,
    line width = +0pt 1,
  },
  setup code = {
    % Control points
    \pgfarrowssetbackend{0pt}
    \pgfarrowssetvisualbackend{\pgfarrowinset}
    \pgfarrowssetlineend{\pgfarrowinset}
    \ifpgfarrowreversed
      \pgfarrowssetlineend{\pgfarrowlength}
    \fi
    \pgfarrowssettipend{\pgfarrowlength}
    % Convex hull
    \pgfarrowshullpoint{\pgfarrowlength}{0pt}
    \pgfarrowsupperhullpoint{0pt}{.5\pgfarrowwidth}
    \pgfarrowshullpoint{\pgfarrowinset}{0pt}
    % The following are needed in the code:
    \pgfarrowssavethe\pgfarrowinset
    \pgfarrowssavethe\pgfarrowlinewidth
    \pgfarrowssavethe\pgfarrowlength
    \pgfarrowssavethe\pgfarrowwidth
  },
  drawing code = {
    \pgfsetdash{}{+0pt}
    \ifdim\pgfarrowlinewidth=\pgflinewidth\else\pgfsetlinewidth{+\pgfarrowlinewidth}\fi
    \pgfpathmoveto{\pgfqpoint{\pgfarrowlength}{0pt}}
    \pgfpathlineto{\pgfqpoint{0pt}{.5\pgfarrowwidth}}
    \pgfpathlineto{\pgfqpoint{\pgfarrowinset}{0pt}}
    \pgfpathlineto{\pgfqpoint{0pt}{-.5\pgfarrowwidth}}
    \pgfpathclose
    \ifpgfarrowopen
      \pgfusepathqstroke
    \else
      \ifdim\pgfarrowlinewidth>0pt\pgfusepathqfillstroke\else\pgfusepathqfill\fi
    \fi
  },
  parameters = {
    \the\pgfarrowlinewidth,%
    \the\pgfarrowlength,%
    \the\pgfarrowwidth,%
    \the\pgfarrowinset,%
    \ifpgfarrowopen o\fi%
  },
}


% For correcting minipage width in stretched nodes
\newdimen\ipeminipagewidth
\def\ipestretchwidth#1{%
  \pgfmathsetlength{\ipeminipagewidth}{#1/\ipenodestretch}}

\tikzstyle{ipe import} = [
  % General ipe defaults
  x=1bp, y=1bp,
%
  % Nodes
  ipe node stretch/.store in=\ipenodestretch,
  ipe stretch normal/.style={ipe node stretch=1},
  ipe stretch normal,
  ipe node/.style={
    anchor=base west, inner sep=0, outer sep=0, scale=\ipenodestretch
  },
%
  % Use a special key for the mark scale, so that the default can be overriden.
  % (This doesn't happen with the scale= key; those accumulate.)
  ipe mark scale/.store in=\ipemarkscale,
%
  ipe mark tiny/.style={ipe mark scale=1.1},
  ipe mark small/.style={ipe mark scale=2},
  ipe mark normal/.style={ipe mark scale=3},
  ipe mark large/.style={ipe mark scale=5},
%
  ipe mark normal, % Set default
%
  ipe circle/.pic={
    \draw[line width=0.2*\ipemarkscale]
      (0,0) circle[radius=0.5*\ipemarkscale];
    \coordinate () at (0,0);
  },
  ipe disk/.pic={
    \fill (0,0) circle[radius=0.6*\ipemarkscale];
    \coordinate () at (0,0);
  },
  ipe fdisk/.pic={
    \filldraw[line width=0.2*\ipemarkscale]
      (0,0) circle[radius=0.5*\ipemarkscale];
    \coordinate () at (0,0);
  },
  ipe box/.pic={
    \draw[line width=0.2*\ipemarkscale, line join=miter]
      (-.5*\ipemarkscale,-.5*\ipemarkscale) rectangle
      ( .5*\ipemarkscale, .5*\ipemarkscale);
    \coordinate () at (0,0);
  },
  ipe square/.pic={
    \fill
      (-.6*\ipemarkscale,-.6*\ipemarkscale) rectangle
      ( .6*\ipemarkscale, .6*\ipemarkscale);
    \coordinate () at (0,0);
  },
  ipe fsquare/.pic={
    \filldraw[line width=0.2*\ipemarkscale, line join=miter]
      (-.5*\ipemarkscale,-.5*\ipemarkscale) rectangle
      ( .5*\ipemarkscale, .5*\ipemarkscale);
    \coordinate () at (0,0);
  },
  ipe cross/.pic={
    \draw[line width=0.2*\ipemarkscale, line cap=butt]
      (-.5*\ipemarkscale,-.5*\ipemarkscale) --
      ( .5*\ipemarkscale, .5*\ipemarkscale)
      (-.5*\ipemarkscale, .5*\ipemarkscale) --
      ( .5*\ipemarkscale,-.5*\ipemarkscale);
    \coordinate () at (0,0);
  },
%
  % Arrow sizes (for TikZ arrows)
  /pgf/arrow keys/.cd,
  ipe arrow normal/.style={scale=1},
  ipe arrow tiny/.style={scale=.4},
  ipe arrow small/.style={scale=.7},
  ipe arrow large/.style={scale=1.4},
  ipe arrow normal,
  /tikz/.cd,
%
  % Approximations to ipe arrows
  % Put in a style to allow to reset default scale when "ipe arrow normal" is
  % changed.  I think this is the only way, since all the parameters to arrows
  % are expanded when the tip is declared.
  ipe arrows/.style={
    ipe normal/.tip={
      ipe _pointed[length=1bp, width=.666bp, inset=0bp,
                   quick, ipe arrow normal]},
    ipe pointed/.tip={
      ipe _pointed[length=1bp, width=.666bp, inset=0.2bp,
                   quick, ipe arrow normal]},
    ipe linear/.tip={
      ipe _linear[length = 1bp, width=.666bp,
                  ipe arrow normal, quick]},
    ipe fnormal/.tip={ipe normal[fill=white]},
    ipe fpointed/.tip={ipe pointed[fill=white]},
    ipe double/.tip={ipe normal[] ipe normal},
    ipe fdouble/.tip={ipe fnormal[] ipe fnormal},
    % These should maybe use [bend], but that often looks bad unless it's on an
    % actual arc.
    ipe arc/.tip={ipe normal},
    ipe farc/.tip={ipe fnormal},
    ipe ptarc/.tip={ipe pointed},
    ipe fptarc/.tip={ipe fpointed},
  },
  ipe arrows, % Set default sizes
]

% I'm not sure how to do this in a .style, since the #args get confused.
\tikzset{
  rgb color/.code args={#1=#2}{%
    \definecolor{tempcolor-#1}{rgb}{#2}%
    \tikzset{#1=tempcolor-#1}%
  },
}

\makeatother

\endinput



\newcommand{\bo}{\omega}
%\newcommand{\KL}{\text{KL}}
\newcommand{\train}{\text{train}}
\newcommand{\D}{\mathcal{D}}
\newcommand{\softmax}{\text{Softmax}}

\newcommand{\logsumexp}{\text{log-sum-exp}}

\newcommand{\vfe}{\mathcal{F}_{\text{VI}}}
\newcommand{\snr}{\text{SNR}}

\newcommand{\R}{\mathbb{R}}
\newcommand{\N}{\mathcal{N}}
\newcommand{\cL}{\mathcal{L}}
\newcommand{\cO}{\mathcal{O}}
\newcommand{\svert}{~|~}
\newcommand{\td}{\text{d}}
\newcommand{\f}{\mathbf{f}}
\newcommand{\x}{\mathbf{x}}
\newcommand{\Bb}{\mathbf{b}}
%\newcommand{\sBb}{\mathtt{b}}
\newcommand{\sBb}{\mathtt{z}}
\newcommand{\bx}{\overline{\x}}
\newcommand{\bb}{\overline{b}}
\newcommand{\y}{\mathbf{y}}
\newcommand{\z}{\mathbf{z}}
\newcommand{\bu}{\mathbf{u}}
\newcommand{\bv}{\mathbf{v}}
\newcommand{\bV}{\mathbf{V}}
\newcommand{\bk}{\mathbf{k}}
\newcommand{\w}{\mathbf{w}}
\newcommand{\W}{\mathbf{W}}
\newcommand{\ba}{\mathbf{a}}
\newcommand{\m}{\mathbf{m}}
\newcommand{\ls}{\mathbf{l}}
\newcommand{\bL}{\mathbf{L}}
\newcommand{\A}{\mathbf{A}}
\newcommand{\X}{\mathbf{X}}
\newcommand{\Z}{\mathbf{Z}}
\newcommand{\BS}{\mathbf{S}}
\newcommand{\Y}{\mathbf{Y}}
\newcommand{\F}{\mathbf{F}}
%\newcommand{\I}{\mathbf{I}}
\newcommand{\M}{\mathbf{M}}
\newcommand{\bp}{\overline{\p}}
\newcommand{\bz}{\mathbf{0}}
\newcommand{\bepsilon}{\text{\boldmath$\epsilon$}}
\newcommand{\bgamma}{\text{\boldmath$\gamma$}}
\newcommand{\s}{\mathbf{s}}
\newcommand{\Unif}{\text{Unif}}
\newcommand{\boh}{\widehat{\text{\boldmath$\omega$}}}
\newcommand{\bsigma}{\text{\boldmath$\sigma$}}
\newcommand{\bSigma}{\text{\boldmath$\Sigma$}}
\newcommand{\bmu}{\text{\boldmath$\mu$}}
\newcommand{\bphi}{\text{\boldmath$\phi$}}
\newcommand{\Kh}{\widehat{\mathbf{K}}}
\newcommand{\tr}{\text{tr}}
\newcommand{\tdet}{\text{det}}
% \newcommand{\KL}{\text{KL}}
\newcommand{\ind}{\mathds{1}}
\newcommand{\bc}{\mathbf{c}}
\newcommand{\reg}{\eta}
\newcommand{\weightdecay}{\lambda}
\newcommand{\h}{\mathbf{h}}

% variables
\newcommand{\mparam}{\bm{\theta}}	% model param
\newcommand{\vparam}{\bm{\phi}}	% variational param

% gradient approximation part
\newcommand{\hparam}{\bm{\varphi}}
\newcommand{\Xb}{\mathbb{X}}
\newcommand{\hgrad}{\overline{\nabla_{\x} \h}}
\newcommand{\Hmatrix}{\mathbf{H}}
\newcommand{\Grad}{\mathbf{G}}
\newcommand{\g}{\bm{g}}
\newcommand{\noise}{\bm{\epsilon}}


\newcommand{\mx}{\vm_\vx}
\newcommand{\my}{\vm_\vy}
\newcommand{\covmat}{\boldsymbol{\Sigma}}
\newcommand{\covx}{\boldsymbol{\Sigma}_{\vx\vx}}
\newcommand{\covy}{\boldsymbol{\Sigma}_{\vy\vy}}
\newcommand{\covxy}{\boldsymbol{\Sigma}_{\vx\vy}}
\newcommand{\covyx}{\boldsymbol{\Sigma}_{\vy\vx}}

\newcommand{\K}{\mathbf{K}}

\newcommand{\questionref}[1]{\Cref{#1} -- \nameref{#1}}

\newcommand{\lb}{\mathcal{L}}
\newcommand{\sumn}{\sum_{n=1}^N}

\newcommand{\dA}{\epsdice{1}}
\newcommand{\dB}{\epsdice{2}}
\newcommand{\dC}{\epsdice{3}}
\newcommand{\dD}{\epsdice{4}}
\newcommand{\dE}{\epsdice{5}}
\newcommand{\dF}{\epsdice{6}}

\theoremstyle{definition}
\newtheorem{question}{Question}

\newcommand{\courseprobstats}{\texttt{50008} \textit{Probability \& Statistics}}


\title{70015 Mathematics for Machine Learning: Exercises}
\author{Mark van der Wilk, Yingzhen Li\footnote{Many thanks to teaching assistants Carles Balsells Rodas, and Alex Spies for their solutions and improvements to the document.} \\ \texttt{\{m.vdwilk,yingzhen.li\}@imperial.ac.uk}}



\begin{document}
\maketitle
\tableofcontents



% \section{Background material}\todo{Adjust.}
% You will be expected to have a \emph{firm} understanding of Mathematics for Machine Learning. In the explanations, I will be manipulating probabilities and expectations freely, as discussed in Mathematics for Machine Learning. If steps are difficult, I encourage you to raise this on the course EdStem page, or during a Q\&A session.

% \begin{itemize}
% \item Basic probability: sample spaces, disjoint events (summation of probabilities), independent events (multiplication of probabilities). See \citet{walpole2012probability}, Ch2 (Imperial Library, or search Google for reading options).
% \item Probability densities. See \citet{mml} \S 6.2.
% \item Sum, product \& Bayes' rules. See \citet{mml} \S 6.3.
% \item Unconstrained continuous optimisation. See \citet{mml} \S 7.1.
% \item Linear algebra and matrix decompositions. See \citet{mml} ch 4 (and Chs 2 and 3 for basics).
% \item A familiarity with linear basis-function regression. See \citet{mml} ch 9.
% \end{itemize}


\section{Notation}
\subsection{Sets}
Throughout this course, we will be using some standard mathematical notation which may be unfamiliar to some. It's ultimately not that special or even crucial to the overall argument, but it is compact (which is practical), and it helps somewhat with practising with expressing things mathematically. Wikipedia has good definitions on these things too.
\begin{itemize}
\item Notation referring to sets of numbers, e.g.~the natural numbers $\mathbb N = \{0, 1, 2, \dots\}$, integers $\mathbb Z = \{\dots, -2, -1, 0, 1, 2, \dots\}$, or real numbers $\mathbb R$.
\item Vectors are sets containing $n$ of some type of object, like reals. We denote the set of all such sets using a superscript notation. For example, all $n$-dimensional vectors becomes $\mathbb R^n$.
\item With $x \in \mathcal S$ we denote that $x$ is an element of the set $\mathcal S$. This allows us to specify that a variable comes from a particular set (or, has a particular type), e.g.~$x \in \Reals^D$.
\item We sometimes use ``set builder'' notation. We did this informally above when defining $\mathbb N$! Usually this works by specifying elements with some property, e.g.~$\mathbf S = \{2n \,|\, n \in \mathbb N\}$, which means ``all the elements 2n such that $n$ is a natural number''. This creates the set of all even positive whole numbers.
\item We denote the union of two sets (the set with all elements that are in either set or both) as $A \cup B$. With set-builder notation this is $A \cup B = \{x\,|\,x\in A \vee x\in B\}$, where $\vee$ means ``or''.
\item We denote the intersection of two sets (the set of all elements that are in both stes) as $A \cap B = \{x\,|\,x\in A \wedge x\in B\}$.
\item For intervals of real numbers, we use brackets, $[,]$, to denote the elements in the set which are "greater than or equal to" and "less than or equal to" an element, respectively. We use parentheses, $(,)$ to denote a strict lower bound or upper bound on the set, respectively. E.g. $[1,5)$ is equivalent to $1 \leq x < 5, x\in\mathbb{R}$.
\item We use the symbol $\neg$ to denote the complement of a set. Given a set containing all elements under consideration $\Omega$, $\neg A$ contains all elements of $\Omega$ that are not in $A$, i.e.~$\neg A = \{x \in \Omega | x \notin A\}$. We can also denote this as $\neg A = \Omega\backslash A$.
\end{itemize}

\subsection{Probabilities}
In this course we will use the notation for probabilities that is common in machine learning. The main advantage is that this notation is shorter, although it does leave certain things implicit. We include this to reduce confusion.

Consider a probability space $(\Omega, \mathcal E, \mathbb P)$ with sample space $\Omega$ (all possible outcomes of a random procedure), event space $\mathcal E$ (the set of all sets of outcomes that we assign a probability to), and probability function $\mathbb P : \mathcal E \to [0, 1]$ (a function that assigns a probability to an event), with a random variable $X: \Omega \to \Reals^D$.
\begin{itemize}
\item With $\prob{E}$ we denote the probability of an event $E \in \mathcal E$, where $E$ is a set of outcomes.
\item Following the usual convention, we use the same notation when considering random variables, e.g.~$\prob{X < 2}$ is short for $\prob{\{s \in \Omega : X(s) < 2\}}$ (see \S6.1 in \courseprobstats).
\item We usually work directly with random variables, and specify all properties using a probability mass function (pmf) or probability density function (pdf). For a specific outcome of the random variable $\alpha$, we write:
\begin{align}
    &\prob{X=\alpha} = p_X(\alpha) && \text{for a pmf } p_X(\cdot) \,, \\
    &\prob{X \in [a, b]} = \int_a^b p_X(\alpha) d\alpha && \text{for a pdf $p_X(\cdot)$ with $\alpha \in \Reals$} \,, \\
    &\prob{X \in A} = \int_A p_X(\alpha) d\alpha && \text{for a pdf $p_X(\cdot)$ with $\alpha \in \Reals^D$}
    \,.
\end{align}
\item Sometimes we may write vectors in boldface, i.e.~$\mathbf x \in \Reals^D$. We won't always though, so keep track of how we define variables!
\item We generally denote outcomes of random variables without referring explicitly to the random variable itself. For example, when we refer to an outcome $\vx$, we implicitly know there is a random variable that can take this value. We usually denote this as the capital, for example here $X$.
\item Sometimes we abuse notation, and drop the random variable when denoting distributions when the argument of the function identifies it, e.g.~$p(\vx) = p_X(\vx)$.
\item If we want to be explicit about the random variable that we are evaluating the density/mass of, I will write e.g.~$p_{X,Y}(\vx,\vy) = p_{X|Y}(\vx|\vy)p_Y(\vy)$.
\item Expectations can be denoted in two ways:
\begin{align}
    &\Exp{X}{f(X)} && \text{to emphasise that X is random, if it is clear what its distribution is}\,, \\
    &\Exp{p(\vx)}{f(\vx)} && \text{to emphasise that we will be integrating over the distribution $p(\vx)$} \,.
\end{align}
In both cases this corresponds to the integral $\int p(\vx) f(\vx) \calcd\vx$.
\item Often, densities and pmfs can be discussed in exactly the same way, if we think of the density of a discrete RV as a sum of delta functions. I.e.~$p(\vx) = \sum_{o} \delta(\vx - \vx_o) p_o$, where $\{\vx_o\}$ is the set of discrete possible outcomes that $X$ can take, and $p_o$ are their corresponding probabilities. This allows us to write an expectation as an integral, regardless of whether the RV is continuous or discrete, because for discrete RVs we get:
\begin{align}
\Exp{p(\vx)}{f(\vx)} = \int p(\vx) f(\vx) \calcd\vx = \int \sum_o \delta(\vx - \vx_o) p_o f(\vx) \calcd\vx = \sum_o f(\vx_o)p_o \,.
\end{align}
(A delta function has the property that $\int_A \delta(\vx) \calcd\vx$ is 1 if $0 \in A$, and 0 otherwise. Linearity of integrals still holds. It can often be seen as the limit of a Gaussian distribution with zero variance.)
\end{itemize}


\section{Formula Sheet}
\begin{itemize}
\item Gaussian probability density function (pdf) with input $\vx \in \Reals^D$, denoted as  $\NormDist{\vx; \vmu, \covmat}$ is
\begin{align}
  p(\vx) = \NormDist{\vx; \vmu, \covmat} = (2\pi)^{-\frac{D}{2}}\detbar{\mathbf{\Sigma}}^{-\frac{1}{2}}\exp\left(-\frac{1}{2}(\vx - \vmu)\transpose\mathbf{\Sigma}^{-1}(\vx-\vmu)\right) \,.
\end{align}
\item For a joint Gaussian density
\begin{align}
\p{\begin{bmatrix}\vx \\ \vy\end{bmatrix}} = \NormDist{\begin{bmatrix}\vx \\ \vy\end{bmatrix}; \begin{bmatrix}\mx \\ \my\end{bmatrix}, \begin{bmatrix}\covx & \covxy \\ \covyx & \covy\end{bmatrix}} \label{eq:gauss-cond-joint} \,,
\end{align}
we have the conditional density
\begin{align}
\p{\vx\given\vy} = \NormDist{\vx; \quad\mx + \covxy\covy\inv(\vy-\my), \quad\covx - \covxy\covy\inv\covyx} \label{eq:gauss-cond} \,.
\end{align}
\end{itemize}





\section{Warm-up Exercises}
To start, here are some exercises which test knowledge which is assumed in the course.

\subsection{Probability Theory}
We assume that you are familiar with probability theory up to the Computing 2nd year \courseprobstats{} course. Here are some questions to serve as a refresher. Students who are not familiar with this background should refer to the notes of \courseprobstats{} or relevant chapters of \citep{mml}. \textbf{We recommend you look at these questions when/before the course starts}. If you need a refresher, or if you do not know the notation, refer to the \courseprobstats{} notes, or discuss with a TA.

\begin{question}[Set Theory and Probability]
\label{q:setsprob}
Using the three axioms of probability show that
\begin{enumerate}[label=\alph*.]
    \item Write down the sample space of a dice. In your notation, use the set A to denote the event of a 3 or 4 occurring. What is the complement of $A$, denoted $\neg A$?
    \item For a problem about lengths, we have a sample space $\Omega = [0, 1]$. For $A = (0.3, 0.4]$, what is $\neg A$?
    \item $\prob{\neg  A} = 1 - \prob{A}$
    \item $\prob{\varnothing} = 0$, where $\varnothing$ is the empty set
    \item $0 \leq \prob{A} \leq 1$
    \item $A \subseteq B \implies \prob{A} \leq \prob{B}$
    
    \textit{Hint:} Consider the following definition. $B\backslash  A = \{x\in B: x\notin A\}$
    \item $\prob{A\cup B} = \prob{A} + \prob{B} - \prob{A\cap B}$
    \item (\textbf{*}) if $\{A_i\}_{i=1}^\infty \subseteq \Omega \text{ and } A_i \subseteq A_{i+1} \forall i$ then:
\[
P\left(\bigcup_{i=1}^{\infty} A_{i}\right) = \lim_{i\xrightarrow{}\infty} \prob{A_i}
\]
\textit{Hint:} Use axiom 3. \textbf{*}: The emphasis of this course isn't on these kinds of details, even though this should be doable with 1st-year calculus.
\item For two mutually exclusive events $A, B$, what is $\prob{A \cup B}$?
\end{enumerate}
See \citet[\S6.1.2]{mml} for a general overview, and \S4, \S\S5.1-5.4 of \courseprobstats{} for more details.
\end{question}

\begin{question}[Independent events] \textbf{Independent events don't come up as much as independent random variables, so it's ok to just follow this answer, rather than spending lots of time on it.}
When tossing two coins (where we care about the order), we have a sample space $\Omega = \{HH, HT, TH, TT\}$.
\begin{enumerate}[label=\alph*.]
    \item What outcomes are contained in the event that corresponds to the the first coin being heads? We denote the event $E_{1H}$, and others similarly.
    \item If you assume that all outcomes have equal probability, show that $E_{1H}$ and $E_{2T}$ are independent.
    \item If you assume that $E_{1H}$ and $E_{2H}$ are independent and 0.5 each, show that all outomes must have equal probability.
\end{enumerate}
See \S5.3.3 in \courseprobstats{}.
\end{question}

\begin{question}[Random Variables]
\label{q:rv}
Consider throwing two fair dice.
\begin{enumerate}[label=\alph*.]
    \item What is the sample space for all outcomes that you can get from throwing two dice? We specify the probability of each outcome to be the same.
    \item Define two random variables $A,B$ which map the outcome to the face value on each die respectively. Find the probability mass function for $A$ from the probability on outcomes. The answer will work from the definition of a random variable, but you will probably intuitively get the right answer as well.
    \item Show that $A$ and $B$ are independent.
    \item Define the random variable $C = A + B$. Derive the probability mass function of $C$.
\end{enumerate}
See \S6 of \courseprobstats{}.
\end{question}


\begin{question}[Continuous Random Variables]
Consider the random variable $X$ with a probability density $p(x) = C\cdot x$ when $x \in [0, 1]$ and $0$ elsewhere.
\begin{enumerate}[label=\alph*.]
    \item Calculate $C$.
    \item Calculate $\prob{0.3 \leq X \leq 0.75}$.
    \item Calculate $\prob{X \in [0.3, 0.75] \cup [0.8, 0.9]}$.
    \item Calculate $\Exp{X}{X}$, $\Exp{X}{X^2}$, $\Var{X}{X}$.
\end{enumerate}
Check your answers by performing numerical integration, e.g.~in Python.

See \S6.3, \S7 of \courseprobstats{} or \citet[\S 6.2.2]{mml}.
\end{question}


\begin{question}[Joint Discrete Random Variables]
Consider two random variables $A,C$, where $A$ is the outcome of one die, and $C$ gives the sum of $A$ and the sum of another die $B$.
\begin{enumerate}[label=\alph*.]
    \item From intuition, write a table of $\prob{C=c|A=a}$, which we use to denote the probability of $C$ taking the value $c$, if we know that $A$ has taken the value $a$.
    \item Write a table of $\prob{C=c, A=a}$. To help you think it through, consider a tree of outcomes that can occur. This helps illustrate independence between outcomes, which helps you figure out when you can multiply probabilities.
    \item From the values in the table $\prob{C=c, A=a}$ find $\prob{2 \leq C \leq 4}$ and $\prob{2 \leq C \leq 4, 2 \leq A \leq 4}$.
\end{enumerate}
We will cover conditional probability more later, but for now just think it through.
\end{question}

\begin{question}[Multivariate Integration]
Consider two continuous random variables $X,Y$ with joint density $p(x, y) = C\cdot (x^2 + xy)$ when $x \in [0, 1]$ and $y \in [0, 1]$, and $0$ elsewhere.
\begin{enumerate}[label=\alph*.]
    \item Find $C$.
    \item Find $\prob{0.3 \leq X \leq 0.5}$.
    \item Find $\prob{X < Y}$. Perform the integration twice in both orders, once integrating over x first, once by integrating over y first.
    \item \textbf{Bonus:} Convince yourself that you know how to do this for $p(x, y, z) = C\cdot (x^2 + xyz)$ as well.
\end{enumerate}
Check your answers by performing numerical integration, e.g.~in Python.
\end{question}

\begin{question}[Statistics Terminology] Recall the following statistical terminology.
\label{q:stats-term}
\begin{enumerate}[label=\alph*.]
    \item What is a statistic?
    \item What is an estimator?
    \item What is a consistent estimator?
    \item What is a sample?
\end{enumerate}
\end{question}

%%%%%%%%%%% Yingzhen's Linear Algebra warm-up questions %%%%%%%%%%

\subsection{Linear Algebra}

\begin{question}[Dot product]
\label{q:dot_product}
Compute $\x^\top \y$ where $\x = (1, -2, 5, -1)^\top$ and $\y = (0, 4, -3, 7)^\top$.
\end{question}

\begin{question}[Matrix product]
\label{q:matrix_product}
Compute $\y = A\x$ as well as the $\ell_2$ norm of $\x$ and $\y$, where
\begin{equation*}
A = \begin{pmatrix}
-1 & 4 & 7 & 2 \\
3 & -2 & -1 & 0 \\
5 & 3 & 0 & -1
\end{pmatrix}, \quad \x = (-3, 2, 1, 3)^\top.
\end{equation*}
\end{question}

\begin{question}[Basis]
\label{q:basis}
Which of the following set of vectors are basis for $\mathbb{R}^2$?
\begin{enumerate}
    \item $\{(1, 1), (1, 0) \}$
    \item $\{(2, 4), (3, -1) \}$
    \item $\{(1, -1), (0, 2), (2, 1) \}$
    \item $\{(2, -1), (-2, 1) \}$
    \item $\{(0, 3) \}$
\end{enumerate}
\end{question}

\begin{question}[Span of vectors] 
\label{q:span}
Which of the following points are within the span of $\{(-1, 0, 2), (3, 1, 0) \}$?
\begin{enumerate}
    \item $(0, 1, 1)$
    \item $(1, 1, 4)$
    \item $(2, 1, 1)$
    \item $(-3, 4, 2)$
    \item $(0, 0, 0)$
\end{enumerate}
\end{question}

\begin{question}[Rotation matrix in $\mathbb{R}^2$]
\label{q:rotation_matrix}
What is the $2 \times 2$ matrix that rotates all the non-zero vectors in $\mathbb{R}^2$ by $45^{\circ}$ counter-clockwise?

\end{question}

\begin{question}[Linear equations]
\label{q:linear_equations}
Given the following system of linear equations:
\begin{equation*}
\begin{aligned}
    x + 2y &= 2 \\
    3x + 2y + 4z &= 5  \\
    -2x + y - 2z &= -1
\end{aligned}
\end{equation*}
Answer the following questions:
\begin{itemize}
    \item[a] Writing this system in a matrix form $A\x = \mathbf{b}$ with $\x = (x, y, z)^\top$. What are $A$ and $\mathbf{b}$?
    \item[b] Solve this system, or show that the solution does not exist.
    \item[c] What is the rank of $A$?
\end{itemize}
\end{question}

\begin{question}[Eigen decomposition]
\label{q:eigen_decomp}
Consider a matrix $A \in \mathbb{R}^{d \times d}$ and assume it has an eigen decomposition of $A = Q \Lambda Q^{-1}$ where $\Lambda = \text{diag}(\lambda_1, ..., \lambda_d)$. When $A$ is symmetric we also have $Q^{-1} = Q^\top$. Answer the following questions:
\begin{itemize}
    \item[a.] If $A$ is symmetric, show that $\x^\top A \x \geq 0$ for any $\x \in \mathbb{R}^{d \times 1}$ if and only if $\lambda_i \geq 0$ for all $i = 1,..., d$.
    \item[b.] Show that $Tr(A) = \sum_{i=1}^d \lambda_i$ where $Tr(A)$ is the trace of $A$.
    \item[c.] Show that $det(A) = \prod_{i=1}^d \lambda_i$ where $det(A)$ is the determinant of $A$.
    \item[d.] Why an entry $\lambda_i$ in the diagonal matrix $\Lambda$ is one of the solutions for the equation $A\bm{q} = \lambda \bm{q}$, $\bm{q} \neq \bm{0}$?
\end{itemize}
\end{question}



\section{Lecture 1: Probability, Vectors, Differentiation}
\begin{question}[Vector notation]
We define the probability density on the vector $\vx \in \Reals^3$ with all elements $0 \leq x_k \leq 1$ as
\begin{align}
p(\vx) = \frac{1}{C} (x_1^2 + x_1x_2 + x_2^2 + 2x_2x_3) \,.
\end{align}
Put this into notation that only uses $\vx$ as a single whole vector.
\end{question}

\begin{question}[Noise conditional independence]
Consider the probability of the data in linear regression, for a fixed setting of the parameters $\vtheta$ and given inputs $\mat X \in \Reals^{N\times D}$ where $\mat X = \{\vx_1, \dots, \vx_N\}$:
\begin{align}
p(\vy|\vtheta,\mat X) = \NormDist{\vy; \vtheta\transpose\vx, \sigma^2 \eye}
\end{align}
Show that all $y_n$s are independent, for a fixed setting of the parameters $\vtheta$ and given inputs $\mat X$.
\end{question}

\begin{question}[Maximum likelihood revision] For a Gaussian distribution with mean $\mu$ and variance $\sigma^2$.
\begin{enumerate}[label=\alph*.]
\item Derive the probability distribution for $N$ iid draws.
\item Derive the maximum likelihood estimator for the mean $\mu$.
\end{enumerate}
\end{question}

\begin{question}[Maximum likelihood and minimum loss]
Show that the solution to the Maximum Likelihood estimator for linear regression is the same as the minimum squared loss estimator.
\end{question}

\begin{question}[MML 5.1-5.3]
This is revision. Compute the derivatives for w.r.t.~$x$ for
\begin{enumerate}[label=\alph*.]
\item $f(x) = \log (x^4) \sin (x^3)$
\item $f(x) = (1 + \exp(-x))^{-1}$
\item $f(x) = \exp\left(-\frac{(x-\mu)^2}{2\sigma^2}\right)$
\end{enumerate}
\end{question}

\section{Lecture 2: Vector Differentiation}
\begin{question}[Circle]
\label{q:circle}
Consider a vector function $\vx(t) = \begin{bmatrix}\cos t & \sin t\end{bmatrix}\transpose$.
\begin{enumerate}[label=\alph*.]
\item Draw the set of points that this function passes through.
\item To build intuition, draw the velocity vector at a few points by considering the direction that the point moves in.
\item Find the derivative $\calcd\vx / \calcd t$. Draw this vector for some point t.
\end{enumerate}
\end{question}

\begin{question}[Index notation] Turn the following matrix-vector expressions into index notation:
\begin{multicols}{2}
\begin{enumerate}[label=\alph*.]
\item $\mat A \mat B \mat C \vx$
\item $\Tr(\mat A)$
\item $\Tr(\mat A \mat B)$
\item $\vy\transpose \mat A\transpose \vx$
\end{enumerate}
\end{multicols}
Turn the following index expressions back to matrix-vector notation:
\begin{multicols}{2}
\begin{enumerate}[label=\alph*.]
\item $\sum_{ijk} A_{ij}B_{jk}C_{ki}$
\item $b_i + \sum_j A_{ij}b_j$
\item $x_ix_j$
\item $\sum_j \delta_{ij}a_j$
\end{enumerate}
\end{multicols}
\end{question}

\begin{question}[Index notation proofs]
Using index notation, show that
\begin{enumerate}
\item $\vx\transpose \mat A\vy = \vy\transpose \mat A\vx$ if $\mat A$ is symmetric, i.e.~$\mat A = \mat A\transpose$.
\item $\vx\transpose\vy = \Tr(\vx\transpose\vy) = \Tr(\vy\transpose\vx)$, for $\vx,\vy\in\Reals^D$.
\item $\Tr(\mat A\mat B\mat C) = \Tr(\mat C\mat A\mat B)$.
\end{enumerate}
\end{question}

\begin{question}[MML 5.5-5.6]
First find the dimensions, then the Jacobian. It's probably easiest here to use index notation.
\begin{enumerate}[label=\alph*.]
\item $f(\vx) = \sin(x_1)\cos(x_2)$, find $\calcd f/\calcd\vx$.
\item $f(\vx) = \vx\transpose\vy$, find $\calcd f/\calcd\vx$.
\item $f(\vx) = \vx\vx\transpose$, find $\calcd f/\calcd\vx$.
\item $f(\vt) = \sin(\log(\vt\transpose\vt))$, find $\calcd f/\calcd\vt$.
\item $f(\mat X) = \Tr(\mat A \mat X \mat B)$ for $\mat A \in \Reals^{D\times E}$, $\mat X \in \Reals^{E\times F}$, $\mat B \in \Reals^{F\times D}$, find $\calcd f/\calcd\mat X$.
\end{enumerate}
\end{question}


\begin{question}[MML 5.7-5.8: Chain rule]
Comupte the derivatives $\calcd f /\calcd\vx$ of the following functions.
\begin{itemize}
\item First, write out the chain rule for the given decomposition.
\item Give the shapes of intermediate results, and make clear which dimension(s) will be summed over.
\item Provide expressions for the derivatives, and describe your steps in detail. Providing an expression means specifying everything up to the point where you could implement it.
\item Give the results in vector notation if you can. 
\end{itemize}
\begin{enumerate}[label=\alph*.]
\item $f(z) = \log(1+z), \qquad z = \vx\transpose\vx, \qquad \vx \in \Reals^D$.
\item $f(\vz) = \sin(\vz), \qquad \vz = \mat A\vx + \vb, \qquad \mat A \in \Reals^{E\times D}$. What sizes are $\vx$ and $\vb$?
\item $f(z) = \exp(-\frac{1}{2}z), \qquad z = \vy\transpose \mat S\inv \vy, \qquad \vy = \vx -\vmu$.
\item $f(\mat A) = \Tr(\mat A), \qquad \mat A = \vx\vx\transpose + \sigma^2 \mat I$.
\item $\vf(\vz) = \tanh(\vz), \qquad \vz = \mat A \vx + \vb, \qquad \mat A \in \Reals^{M\times N}$.
\item $f(\mat A) = \vx\transpose\mat A\vx, \qquad \mat A = \vx\vx\transpose\,.$
\end{enumerate}
Remember: Generally, scalar functions are applied elementwise to vectors/matrices.
\end{question}



\begin{question}[Hessian of Linear Regression]
\label{q:hessian}
For the stationary point of linear regression, find the Hessian, and prove that it is positive definite, perhaps by making some assumptions. Discuss your assumptions.
\end{question}


\section{Lecture 3: Automatic Differentiation}
\begin{question}[Product rule]
\label{q:autodiff-productrule} Consider the function $f(a, b) = a\cdot b$, where $a = a(x), b = b(x)$, i.e.~unspecified functions of $x$.
\begin{itemize}
\item Show that by following forward mode autodiff, you effectively calculate the product rule.
\item Show that if $a(x) = x, b(x) = x$, which means that the overall function $f(x) = x^2$, the gradient that is computed will be $2x$.
\end{itemize}
(Note from MvdW (autumn 2022): I somewhat messily described this on the board. The question is included here to provide a clearer explanation.)
\end{question}


\begin{question}[Multivariate Autodiff]
\label{q:autodiff} This is a rather big question that should test your understanding of all material in the first three lectures.
Consider the overall function $f(\boldsymbol \ell, \mat X)$ consisting of the parts:
\begin{align}
f &= \vy\transpose \left(\mat K_1 + \mat K_2\right)\inv \vy\,, \\
\mat K_a &= \exp\left(\mat \Lambda_a\right) \,, \\
\mat \Lambda_a &= -\frac{\mat D_a}{2\ell_a^2} \,, \\
\mat D_a &= (\mat X[:, \text{\texttt{None}}, a] - \mat X[\text{\texttt{None}}, :, a])^2 \,,
\end{align}
where we use \texttt{numpy} broadcasting notation in the final equation.
\begin{enumerate}[label=\alph*.]
\item Given $\boldsymbol \ell \in \Reals^2$ and $\mat X \in \Reals^{N\times 2}$, find the shape of all intermediate computations.
\item Draw the computational graph for $f(\boldsymbol\ell, \mat X)$.
\item For forward and reverse mode differentiation, state which intermediate derivatives are computed at each step, and their computational and memory costs.
\end{enumerate}
\end{question}

% Lecture 4: Probabilistic Modelling Principles 
% \section{Lecture 4: Probabilistic Modelling Principles}

\begin{question}[Training translation models]
\label{q:translation}
Imagine you want to train a neural network $T_{\mparam}(\cdot)$ to translate French words to English words. Assume you are given a dataset $\mathcal{D} = \{(f_n, e_n) \}_{n=1}^N$ where $f_n$ is a French word and $e_n$ is an English word. Suppose the vocabulary of French and English is $\mathcal{F}$ and $\mathcal{E}$, respectively.
%
\begin{itemize}
    \item[a.] Assuming a probabilistic model $p(e | T_{\mparam}(f))$, which distribution would you choose for this model?
    \item[b.] Continuing a), what is the corresponding MLE objective? 
\end{itemize}
%
\end{question}

\begin{question}[Clustering]
\label{q:clustering_gmm}
We consider a clustering task where given a dataset $\mathcal{D} = \{x_1, ...,x_N \}$, we would like to group them into $K$ clusters. The model we will use here is a Gaussian mixture model:
$$\text{GMM:} \quad p(x | \mparam) = \sum_{k=1}^K \pi_k \mathcal{N}(x; \mu_k, \sigma^2), \quad \mparam = \{ \pi_k, \mu_k, \}_{k=1}^K, \sigma^2.$$
%
\begin{itemize}
    \item[a.] What is the MLE objective for this clustering task?
    \item[b.] Derive the gradient of the MLE objective w.r.t.~$\mu_k$. What is the fixed-point equation for finding the optimal $\{ \mu_k \}$ parameters?
\end{itemize}

\end{question}


\begin{question}[Geometric interpretation of linear regression]
\label{q:linear_regression_projection}
Consider the following linear regression model:
$$y = \mparam^\top \phi(\x) + \epsilon, \quad \epsilon \sim \mathcal{N}(0, \sigma^2).$$
For a given dataset $\{ (\x_n, y_n) \}_{n=1}^N$, Writing $\bm{\Phi} = (\phi(\x_1), \phi(\x_2), ..., \phi(\x_N))^\top$ and $\y = (y_1, ..., y_N)^\top$, we have the optimal solution satisfies $\mparam^* = (\bm{\Phi}^\top \bm{\Phi})^{-1} \bm{\Phi}^\top \y$. Show that by using the optimal parameter $\mparam^*$, the prediction $\hat{\y} = (\hat{y}_1, ..., \hat{y}_N), \hat{y}_n = (\mparam^*)^\top \phi(\x_n)$ is the projection of $\y$ onto the sub-space spanned by the columns of $\Phi$.

(Hint: consider singular value decomposition.)

\end{question}

% Lecture 5: Gradient Descent Convergence
% \section{Lecture 5: Gradient Descent Convergence}

\newcommand{\BA}{\mathbf A}

\begin{question}[Rayleigh quotient]
\label{q:rayleigh_quotient}
The \emph{Rayleigh quotient} is defined for a symmetric matrix $\BA \in \mathbb{R}^{d \times d}$ and a non-zero vector $\x \in \mathbb{R}^{d \times 1}$:
\begin{equation*}
R(\BA, \x) = \frac{\x^\top \BA \x}{|| \x ||_2^2}, \quad || \x ||_2^2 = \x^\top \x.
\end{equation*}
Show that
$R(\BA, \x) \in [\lambda_{min}(\BA), \lambda_{max}(\BA)].$

This result immediately indicates that $\lambda_{min}(\BA) || \x ||_2^2 \leq \x^\top \BA \x \leq \lambda_{max}(\BA) || \x ||_2^2$, which is used to prove gradient descent convergence.
\end{question}


\begin{question}[Gradient descent with pre-conditioning]
\label{q:pre_conditioned_gd}
Consider the following update rule named \emph{pre-conditioned gradient descent}:
$$\mparam_{t+1} = \mparam_t - \gamma_t \BP_t^{-1} \nabla_{\mparam} L(\mparam_t).$$
Here $\BP_t$ is called \emph{pre-conditioner} at time step $t$. We consider linear regression as an example, and assume constant learning rate and pre-conditioner, i.e., $\gamma_t = \gamma$ and $\BP_t = \BP$ for all $t$. 
%
Show that with an appropriate choice of the pre-conditioner $\BP$, we can achieve a robust selection of the learning rate $\gamma$, i.e., if the selected $\gamma$ works for an initialisation $\mparam_0$, it will also work for all other initialisations.

Hints: you can follow the below steps to solve the question:
\begin{itemize}
    \item[1.] Work out the pre-conditioned gradient descent update in linear regression, and derive $\mparam_t$ as a function of $\mparam_0$, $\gamma$, $\BP$ and the dataset $(\X, \y)$;
    \item[2.] For a given $\BP$, work out the learning rates $\gamma_{min}$ and $\gamma_{max}$ such that pre-conditioned gradient descent converges when $\gamma < \gamma_{min}$, or diverges when $\gamma \geq \gamma_{max}$;
    \item[3.] Select $\BP$ such that $\gamma_{min} = \gamma_{max}$, therefore there exist no interval (like $[\gamma_{min}, \gamma_{max})$) such that convergence depends on initialisation when $\gamma$ falls into such interval.
\end{itemize}
\end{question}


\begin{question}[Momentum gradient descent]
\label{q:momemtum_gd}
Consider the following update rule named \emph{momemtum gradient descent}, with constant learning rate $\gamma$ and momentum step-size $\alpha$:
$$\mparam_{t+1} = \mparam_t - \gamma \nabla_{\mparam} L(\mparam_t) + \alpha \Delta \mparam_t,$$
$$\Delta \mparam_{t+1} = \mparam_{t+1} - \mparam_t, \quad \Delta \mparam_{0} = \bm{0}.$$

Show that solving linear regression using momemtum gradient descent, if converges, converges to $\mparam^* = (\X^\top \X)^{-1} \X^\top \y$.

Hint: follow the below steps and practice your linear algebra skills :)
\begin{enumerate}
    \item Write down the update equations for the parameters $\mparam_t$ and the momentum $\Delta \mparam_t$;
    \item Collect both terms as a long vector $(\mparam_t^\top, \Delta \mparam_t^\top)^\top$, and merge the two linear update equations in step 1 into one ``joint'' linear equation using block matrices;
    \item Apply the analysis techniques for gradient descent convergence for linear regression to show the converged solution (if converges).
\end{enumerate}

\end{question}


\section{Lecture $N$: Multivariate Probability}
\begin{question}[Vector independence] {\color{red}While you can probabily figure this one out already, we will discuss this in more detail later.}
Consider the density on $\vx \in \Reals^4$ with all elements $0 \leq x_k \leq 1$ as
\begin{align}
p(\vx) = \vx\transpose \begin{bmatrix} 0 & 0 & 0 & 0 \\ 1 & 0 & 1 & 0 \\ 0 & 0 & 0 & 0 \\ 1 & 0 & 1 & 0\end{bmatrix} \vx \,.
\end{align}
\begin{enumerate}[label=\alph*.]
\item Rewrite the density in terms of $\tilde\vx = [x_1, x_3, x_2, x_4]\transpose$. Note that you can do this by a substitution $\vx = \mat P \tilde\vx$, where $\mat P$ is a permutation matrix. You will see that you just need to swap the relevant rows and columns of the matrix. However, make sure that you understand the mathematical steps that really show this.
\item Divide up $\vx$ into two sub vectors $\vy = [x_2, x_4]\transpose$ and $\vz = [x_1, x_3]\transpose$. Show that $\vy \ci \vz$, i.e.~that they are independent.
\end{enumerate}
\end{question}








%%%%%%%%%%%%%%%%%%%%%%%%%%%%%%%%%%%



% \section{Lecture 1: Multivariate Probability}
% \begin{question}[Notation]
% \end{question}




\section{Warm-up Exercises Answers}
\subsection{Probability Theory}
\paragraph{\questionref{q:setsprob}}
\begin{enumerate}[label=\alph*.]
    \item We can choose any representation denoting the events, e.g. using abstract symbols $\Omega=\{ \epsdice{1}, \epsdice{2}, \epsdice{3}, \epsdice{4}, \epsdice{5}, \epsdice{6} \}$. Alternatively, we can represent each of the outcomes as a number $\Omega = \{1, 2, 3, 4, 5, 6\}$.
    
    Following the latter notation, $A=\{3, 4\}$, and $A=\{1, 2, 5, 6\}$.

	    
    
	\item Length problem with sample space $\Omega=[0,1]$.
	
	$\neg A = [0, 0.3] \cup (0.4,1]$
    
    \item $P(\neg  A) = 1 - P(A)$
    
    Since $\neg A$ and $A$ are mutually exclusive: $A \cup \neg A = \Omega$ and $A \cap \neg A = \varnothing$.

By combining axiom 2 and 3: $P(A) + P(\neg A) = P(A \cup \neg A) = P(\Omega) = 1$

Thus: $P(\neg A) = 1 - P(A)$ 

    \item $P(\varnothing) = 0$, where $\varnothing$ is the empty set
    
Given the sample space, $\Omega$, its complementary is the empty set $\varnothing$. 

We use property (c) and axiom 2: $ P(\varnothing) = 1 - P(\Omega) = 1 - 1 = 0$.

    \item $0 \leq P(A) \leq 1$
    
We use property (c) and axiom 1. 

Consider an event $A$, where $P(A) \geq 0$ and $P(\neg A) \geq 0$ by axiom 1. 

Then, $P(\neg A) = 1 - P(A) \geq 0 \implies 1 \geq P(A)$.

By joining both inequalities, $0 \leq P(A) \leq 1$.

    \item $A \subseteq B \implies P(A) \leq P(B)$
    
    \textit{Hint:} Consider the following definition. $B\backslash  A = \{x\in B: x\notin A\}$

Assume $A \subseteq B$ and construct $B$ as the union of two disjoint sets: $B = B\backslash A \cup A$.

Then, $B\backslash A \cap A = \varnothing$ by definition of $B\backslash A$. By axiom 1, we have $P(B\backslash A) \geq 0$. 

Use axiom 3: $P(B) = P(B\backslash A) + P(A) \geq P(A) \implies P(A) \leq P(B)$.


    \item $P(A\cup B) = P(A) + P(B) - P(A\cap B)$.

Define the union $(A\cup B)$ in terms of two disjoint sets. $(A \cup B) = A \cup B\backslash A$, where $A \cap B\backslash A = \varnothing$. 

Use axiom 3: $P(A \cup B) = P(A) + P(B\backslash A)$.

To compute $P(B\backslash A)$, we define B in terms of A, and the union of two disjoint sets: $B = (B \cap A) \cup (B\backslash A)$, where $(B \cap A) \cap (B\backslash A) = \varnothing$ by definition. 

Use axiom 3 again: $P(B) = P(B \cap A) + P(B\backslash A) \implies P(B\backslash A) = P(B) - P(B \cap A)$.

Finally: $P(A \cup B) = P(A) + P(B\backslash A) = P(A) + P(B) - P(B \cap A)$.

    \item (\textbf{*}) if $\{A_i\}_{i=1}^\infty \subseteq \Omega \text{ and } A_{i-1} \subseteq A_{i}\quad \forall i>0$ then:
\[
P\left(\bigcup_{i=1}^{\infty} A_{i}\right) = \lim_{i\xrightarrow{}\infty} P(A_i)
\]
\textit{Hint:} Use axiom 3.

Let us define the following: $A := \bigcup_{i=1}^{\infty} A_{i}$. We would like to write $A$ in terms of disjoint sets to use axiom 3.
\begin{align}\label{eq:sets:disjoint-sets}
A_{i-1} \subseteq A_i \quad \forall i > 0 \implies A = \bigcup_{i=1}^{\infty} A_{i}\backslash A_{i-1}
\end{align}
where the expression holds if we have $A_{0} = \varnothing$. We regard \ref{eq:sets:disjoint-sets} as starting with $A_1$ and adding the new information from $A_2, A_3,\dots$ (e.g $A_2 \backslash A_1, A_3 \backslash A_2, \dots$). 
\begin{align}\label{eq:sets:axiom3-on-set}
P(A) = P\left(\bigcup_{i=1}^{\infty} A_{i}\backslash A_{i-1}\right) = \sum_{i=1}^\infty P(A_{i}\backslash A_{i-1})&&\text{(by axiom 3)}\\
P(A) = \sum_{i=1}^\infty P(A_{i}\backslash A_{i-1}) = \lim_{n\xrightarrow{}\infty} \sum_{i=1}^n P(A_{i}\backslash A_{i-1}) &&\text{(the infinite summation is a limit)}
\end{align}
From (f), we have $P(A_i) = P(A_{i}\backslash A_{i-1}) + P(A_{i-1}) \implies  P(A_{i}\backslash A_{i-1}) = P(A_i) - P(A_{i-1})$. Then,
\begin{align}
P(A) = \lim_{n\xrightarrow{}\infty} \sum_{i=1}^n P(A_i) - P(A_{i-1}) = \lim_{n\xrightarrow{}\infty} \bigg(\sum_{i=1}^n P(A_i) - \sum_{i=1}^{n-1} P(A_{i})\bigg) = \lim_{n\xrightarrow{}\infty} P(A_n)
\end{align}
where we used $P(A_0) = P(\varnothing) = 0$ from (d). 

In summary:
\begin{align}
P\left(\bigcup_{i=1}^{\infty} A_{i}\right) = P(A) = \lim_{i\xrightarrow{}\infty} P(A_i)
\end{align}

\end{enumerate}

\paragraph{\questionref{q:rv}}
\begin{enumerate}[label=\alph*.]
\item
We choose to represent the outcomes of two dice as integer tuples:
\begin{align*}
\Omega = \{
&(\dA,\dA), (\dA,\dB), (\dA,\dC), (\dA,\dD), (\dA,\dE), (\dA,\dF),  \\
&(\dB,\dA), (\dB,\dB), (\dB,\dC), (\dB,\dD), (\dB,\dE), (\dB,\dF),  \\
&(\dC,\dA), (\dC,\dB), (\dC,\dC), (\dC,\dD), (\dC,\dE), (\dC,\dF),  \\
&(\dD,\dA), (\dD,\dB), (\dD,\dC), (\dD,\dD), (\dD,\dE), (\dD,\dF),  \\
&(\dE,\dA), (\dE,\dB), (\dE,\dC), (\dE,\dD), (\dE,\dE), (\dE,\dF),  \\
&(\dF,\dA), (\dF,\dB), (\dF,\dC), (\dF,\dD), (\dF,\dE), (\dF,\dF) \}
\end{align*}
\item
We define random variables A and B to be:
\begin{align*}
A(s) =
\begin{cases}
1 &\text{, if } s \in \{ (\dA, \dA), (\dA,\dB), (\dA,\dC), (\dA,\dD), (\dA,\dE), (\dA,\dF) \} \\
2 &\text{, if } s \in \{ (\dB, \dA), (\dB,\dB), (\dB,\dC), (\dB,\dD), (\dB,\dE), (\dB,\dF) \} \\
3 &\text{, if } s \in \{ (\dC, \dA), (\dC,\dB), (\dC,\dC), (\dC,\dD), (\dC,\dE), (\dC,\dF) \} \\
4 &\text{, if } s \in \{ (\dD, \dA), (\dD,\dB), (\dD,\dC), (\dD,\dD), (\dD,\dE), (\dD,\dF) \} \\
5 &\text{, if } s \in \{ (\dE, \dA), (\dE,\dB), (\dE,\dC), (\dE,\dD), (\dE,\dE), (\dE,\dF) \} \\
5 &\text{, if } s \in \{ (\dF, \dA), (\dF,\dB), (\dF,\dC), (\dF,\dD), (\dF,\dE), (\dF,\dF) \}
\end{cases} \\
B(s) =
\begin{cases}
1 &\text{, if } s \in \{ (\dA,\dA), (\dB,\dA), (\dC,\dA), (\dD,\dA), (\dE,\dA), (\dF,\dA) \} \\
2 &\text{, if } s \in \{ (\dA,\dB), (\dB,\dB), (\dC,\dB), (\dD,\dB), (\dE,\dB), (\dF,\dB) \} \\
3 &\text{, if } s \in \{ (\dA,\dC), (\dB,\dC), (\dC,\dC), (\dD,\dC), (\dE,\dC), (\dF,\dC) \} \\
4 &\text{, if } s \in \{ (\dA,\dD), (\dB,\dD), (\dC,\dD), (\dD,\dD), (\dE,\dD), (\dF,\dD) \} \\
5 &\text{, if } s \in \{ (\dA,\dE), (\dB,\dE), (\dC,\dE), (\dD,\dE), (\dE,\dE), (\dF,\dE) \} \\
5 &\text{, if } s \in \{ (\dA,\dF), (\dB,\dF), (\dC,\dF), (\dD,\dF), (\dE,\dF), (\dF,\dF) \}
\end{cases}
\end{align*}
We can find the PMFs by counting the number of occurrences in $\Omega$. For instance:
\begin{align*}
p_A(3) = \frac{|\{ (\dC, \dA), (\dC,\dB), (\dC,\dC), (\dC,\dD), (\dC,\dE), (\dC,\dF) \}|}{|\Omega|} = \frac{6}{36} = \frac{1}{6}
\end{align*}
Repeating this for all outcomes gives us the full PDFs:
\begin{align*}
\begin{split}
p_A(x) = \begin{cases}
\frac{1}{6} &\text{, if } x = 1\\
\frac{1}{6} &\text{, if } x = 2\\
\frac{1}{6} &\text{, if } x = 3\\
\frac{1}{6} &\text{, if } x = 4\\
\frac{1}{6} &\text{, if } x = 5\\
\frac{1}{6} &\text{, if } x = 6\\
0 &\text{, otherwise } \\
\end{cases}
\end{split}\text{, }
\begin{split}
p_B(x) = \begin{cases}
\frac{1}{6} &\text{, if } x = 1\\
\frac{1}{6} &\text{, if } x = 2\\
\frac{1}{6} &\text{, if } x = 3\\
\frac{1}{6} &\text{, if } x = 4\\
\frac{1}{6} &\text{, if } x = 5\\
\frac{1}{6} &\text{, if } x = 6\\
0 &\text{, otherwise } \\
\end{cases}
\end{split}
\end{align*}
\item
To show independence of $A$ and $B$ we must show that $p(A \cap B) = p(A)p(B)$.
We have that all outcomes have equal probability $\frac{1}{|\Omega|} = \frac{1}{36}$ and therefore:
\begin{align*}
p(A \cap B) = \frac{1}{36} = \frac{1}{6} \cdot \frac{1}{6} = p(A) p(B)
\end{align*}
\item
We can define a random variable $C = A + B$
\begin{align*}
C(s) =
\begin{cases}
2 &\text{, if } s \in \{ (\dA, \dA) \} \\
3 &\text{, if } s \in \{ (\dA, \dB), (\dB,\dA) \} \\
4 &\text{, if } s \in \{ (\dA, \dC), (\dB,\dB), (\dC,\dA) \} \\
5 &\text{, if } s \in \{ (\dA, \dD), (\dB,\dC), (\dC,\dB), (\dD,\dA) \} \\
6 &\text{, if } s \in \{ (\dA, \dE), (\dB,\dD), (\dC,\dC), (\dD,\dB), (\dE,\dA) \} \\
7 &\text{, if } s \in \{ (\dA, \dF), (\dB,\dE), (\dC,\dD), (\dD,\dC), (\dE,\dB), (\dF,\dA) \} \\
8 &\text{, if } s \in \{ (\dB, \dF), (\dC,\dE), (\dD,\dD), (\dE,\dC), (\dF,\dB) \} \\
9 &\text{, if } s \in \{ (\dC, \dF), (\dD,\dE), (\dE,\dD), (\dF,\dC) \} \\
10 &\text{, if } s \in \{ (\dD, \dF), (\dE,\dE), (\dF,\dD) \} \\
11 &\text{, if } s \in \{ (\dE, \dF), (\dF,\dE) \} \\
12 &\text{, if } s \in \{ (\dF, \dF) \} \\
\end{cases}
\end{align*}
Then the PDF $p_C$ becomes:
\begin{align*}
p_C(x) = \begin{cases}
\frac{1}{36} = \frac{1}{36} &\text{, if } x = 2\\
\frac{2}{36} = \frac{1}{18} &\text{, if } x = 3\\
\frac{3}{36} = \frac{1}{12} &\text{, if } x = 4\\
\frac{4}{36} = \frac{1}{9}  &\text{, if } x = 5\\
\frac{5}{36} = \frac{5}{36} &\text{, if } x = 6\\
\frac{6}{36} = \frac{1}{6}  &\text{, if } x = 7\\
\frac{5}{36} = \frac{5}{36} &\text{, if } x = 8\\
\frac{3}{36} = \frac{1}{9}  &\text{, if } x = 9\\
\frac{3}{36} = \frac{1}{12} &\text{, if } x = 10\\
\frac{2}{36} = \frac{1}{18} &\text{, if } x = 11\\
\frac{1}{36} = \frac{1}{36} &\text{, if } x = 12\\
0 &\text{, otherwise } \\
\end{cases}
\end{align*}
which can be rewritten in more compact form:
\begin{align*}
p_C(x) = \begin{cases}
\frac{6 - |x-6|}{36} &\text{, if } x = \{ 2, 3, \ldots, 12\}\\
0 &\text{, otherwise } \\
\end{cases}
\end{align*}
\end{enumerate}

\paragraph{\questionref{q:crv}}

\begin{enumerate}[label=\alph*.]
\item
\begin{align*}
1 = \int_{-\infty}^{\infty} p(x) \mathrm d x = \int_0^1 Cx \mathrm d x = \frac{1}{2} C x^2 \Big|_{0}^1 = \frac{1}{2}C = 1 \\
\implies C = 2
\end{align*}
\item
\begin{align*}
\mathbb{P}(0.3 \leq X \leq 0.75) = \int_{0.3}^{0.75} 2x \mathrm d x = x^2 \Big|_{0.3}^{0.75} = 0.75^2 - 0.3^2 = 0.4725
\end{align*}
\item
\begin{align*}
\mathbb{P}(X \in [0.3, 0.75] \cup [0.8, 0.9])
&= \int_{0.3}^{0.75} 2x \mathrm d x + \int_{0.8}^{0.9} 2x \mathrm d x \\
&= x^2 \Big|_{0.3}^{0.75} + x^2 \Big|_{0.8}^{0.9} = 0.75^2 - 0.3^2 + 0.9^2 - 0.8^2 = 0.6425
\end{align*}
\item
\begin{align*}
\mathbb{E}_X[X] &= \int x p(x) \mathrm d x = \int_0^1 2 x^2 \mathrm d x = \frac{2}{3} x^3 \Big|_0^1 = \frac{2}{3} \\
\mathbb{E}_X[X^2] &= \int x^2 p(x) \mathrm d x = \int_0^1 2 x^3 \mathrm d x = \frac{2}{4} x^4 \Big|_0^1 = \frac{1}{2}
\end{align*}
We can derive the following useful identity that generally holds:
\begin{align*}
\mathbb{V}_X[X] &= \mathbb{E}_X[(X - E_X[X])^2] \\
&= \mathbb{E}_X[X^2 - 2X \mathbb{E}_X[X] + \mathbb{E}_X[X]^2 \\
&= \mathbb{E}_X[X^2] - 2 \mathbb{E}_X \mathbb{E}_X[X] + \mathbb{E}_X[X]^2 \\
&= \mathbb{E}[X^2] - \mathbb{E}[X]^2
\end{align*}
Using this fact $\mathbb{V}_X[X] = \mathbb{E}_X[X^2] - (\mathbb{E}_X[X])^2$ we can calculate the variance:
\begin{align*}
\mathbb{V}_X[X] &= \mathbb{E}_X[X^2] - (\mathbb{E}_X[X])^2 = \frac{1}{2} - (\frac{2}{3})^2 = \frac{1}{18} \\
\end{align*}

\end{enumerate}

\paragraph{\questionref{q:jdrv}}

\begin{enumerate}[label=\alph*.]
\item
\begin{tabular}{c|cccccc}
$P(C=c \mid A=a)$ & $a=1$ & $a=2$ & $a=3$ & $a=4$ & $a=5$ & $a=6$ \\
\hline
$c = 2$ & $\frac{1}{6}$ & $0$ & $0$ & $0$ & $0$ & $0$ \\
$c = 3$ & $\frac{1}{6}$ & $\frac{1}{6}$ & $0$ & $0$ & $0$ & $0$ \\
$c = 4$ & $\frac{1}{6}$ & $\frac{1}{6}$ & $\frac{1}{6}$ & $0$ & $0$ & $0$ \\
$c = 5$ & $\frac{1}{6}$ & $\frac{1}{6}$ & $\frac{1}{6}$ & $\frac{1}{6}$ & $0$ & $0$ \\
$c = 6$ & $\frac{1}{6}$ & $\frac{1}{6}$ & $\frac{1}{6}$ & $\frac{1}{6}$ & $\frac{1}{6}$ & $0$ \\
$c = 7$ & $\frac{1}{6}$ & $\frac{1}{6}$ & $\frac{1}{6}$ & $\frac{1}{6}$ & $\frac{1}{6}$ & $\frac{1}{6}$ \\
$c = 8$ & $0$ & $\frac{1}{6}$ & $\frac{1}{6}$ & $\frac{1}{6}$ & $\frac{1}{6}$ & $\frac{1}{6}$ \\
$c = 8$ & $0$ & $0$ & $\frac{1}{6}$ & $\frac{1}{6}$ & $\frac{1}{6}$ & $\frac{1}{6}$ \\
$c = 8$ & $0$ & $0$ & $0$ & $\frac{1}{6}$ & $\frac{1}{6}$ & $\frac{1}{6}$ \\
$c = 8$ & $0$ & $0$ & $0$ & $0$ & $\frac{1}{6}$ & $\frac{1}{6}$ \\
$c = 8$ & $0$ & $0$ & $0$ & $0$ & $0$ & $\frac{1}{6}$ \\
\end{tabular}
\item
\begin{tabular}{c|cccccc}
$P(C=c, A=a)$ & $a=1$ & $a=2$ & $a=3$ & $a=4$ & $a=5$ & $a=6$ \\
\hline
$c = 2$  & $\frac{1}{36}$ & $0$ & $0$ & $0$ & $0$ & $0$ \\
$c = 3$  & $\frac{1}{36}$ & $\frac{1}{36}$ & $0$ & $0$ & $0$ & $0$ \\
$c = 4$  & $\frac{1}{36}$ & $\frac{1}{36}$ & $\frac{1}{36}$ & $0$ & $0$ & $0$ \\
$c = 5$  & $\frac{1}{36}$ & $\frac{1}{36}$ & $\frac{1}{36}$ & $\frac{1}{36}$ & $0$ & $0$ \\
$c = 6$  & $\frac{1}{36}$ & $\frac{1}{36}$ & $\frac{1}{36}$ & $\frac{1}{36}$ & $\frac{1}{36}$ & $0$ \\
$c = 7$  & $\frac{1}{36}$ & $\frac{1}{36}$ & $\frac{1}{36}$ & $\frac{1}{36}$ & $\frac{1}{36}$ & $\frac{1}{36}$ \\
$c = 8$  & $0$ & $\frac{1}{36}$ & $\frac{1}{36}$ & $\frac{1}{36}$ & $\frac{1}{36}$ & $\frac{1}{36}$ \\
$c = 9$  & $0$ & $0$ & $\frac{1}{36}$ & $\frac{1}{36}$ & $\frac{1}{36}$ & $\frac{1}{36}$ \\
$c = 10$ & $0$ & $0$ & $0$ & $\frac{1}{36}$ & $\frac{1}{36}$ & $\frac{1}{36}$ \\
$c = 11$ & $0$ & $0$ & $0$ & $0$ & $\frac{1}{36}$ & $\frac{1}{36}$ \\
$c = 12$ & $0$ & $0$ & $0$ & $0$ & $0$ & $\frac{1}{36}$ \\
\end{tabular}
\item
\begin{align*}
\mathbb{P}(2 \leq C \leq 4) &=
\sum_{c\in\{2, 3, 4\}} \sum_{a} \mathbb{P}(C=c, A=a) \\
&= 6 \cdot \frac{1}{36} + 12 \cdot 0 = \frac{1}{6} \\
\mathbb{P}(2 \leq C \leq 4, 2 \leq A \leq 4) &= \sum_{c\in\{2, 3, 4\}} \sum_{a \in \{ 2, 3, 4\}} \mathbb{P}(C=c, A=a) \\
&= 3 \cdot \frac{1}{36} + 6 \cdot 0 = \frac{1}{12} \\
\end{align*}
\end{enumerate}

\paragraph{\questionref{q:mi}}

\begin{enumerate}[label=\alph*.]
    \item
The probability of the sample space should equal to 1, by the axioms of probability theory. Therefore, if we integrate over all possible outcomes of random variable $X$, from $-\infty$ to $+\infty$, we should obtain 1. We have that $p(x, y)$ is only defined on a small region, namely when $x \in [0, 1]$ and $y \in [0, 1]$. Outside this region the integral evaluates to zero. By the sum rule of integration, this means we can evaluate the integral by only integrating over the region where $p(x, y)$ is non-zero:
\begin{align*}
1 &= \int_{-\infty}^{\infty} p(x, y) \mathrm d x \mathrm d y =  \int_0^1 \int_0^1 C (x^2 + xy) \mathrm d x \mathrm d y \\
&= C \int_0^1 \left( \left( \frac{1}{3} x^3\Big|_{x=0}^1 \right) + \left( \frac{1}{2} y x^2 \Big|_{x=0}^1 \right) \right) \mathrm d y \\
&= C \int_0^1 \left(\frac{1}{3} +\frac{1}{2} y \right) \mathrm d y \\
&= C \left(\frac{1}{3}y \Big|_0^1\right) + C \left(\frac{1}{4} y^2 \Big|_0^1 \right) \\
&= C \left( \frac{1}{3} + \frac{1}{4} \right) = C \frac{7}{12} = 1\\
&\implies C = \frac{12}{7}
\end{align*}
\item
To find $\mathbb{P}(0.3 \leq X \leq 0.5)$, we again integrate over all possible outcomes of random variable $Y$, that is from $-\infty$ to $+\infty$ and only consider outcomes of $X$ by considering range $x \in [0.3, 0.5]$:
\begin{align*}
\mathbb{P}(0.3 \leq X \leq 0.5)
&= \int_{-\infty}^{\infty} \int_{0.3}^{0.5} p(x, y) \mathrm d x \mathrm d y
\end{align*}
Since $p(x, y)$ is only defined on a small region, namely when $x \in [0, 1]$ and $y \in [0, 1]$. Outside this region, we have that $p(x, y){=}0$ evaluates to zero. So, by the sum rule of integration, what remains is the integrate over the region where $p(x, y)$ takes the non-zero value $(Cx^2 + Cxy)$:
\begin{align*}
\mathbb{P}(0.3 \leq X \leq 0.5)
&= \int_{-\infty}^{\infty} \int_{0.3}^{0.5} p(x, y) \mathrm d x \mathrm d y \\
&= \int_{0}^{1} \int_{0.3}^{0.5} p(x, y) \mathrm d x \mathrm d y \\
&= \int_0^1 \int_{0.3}^{0.5} \left(Cx^2 + Cxy\right) \mathrm d x \mathrm d y \\
&= C \int_0^1 \left(\left(\frac{1}{3}x^3 \Big|_{0.3}^{0.5} \right) + \left( \frac{1}{2}x^2 y \Big|_{x=0.3}^{0.5} \right) \right) \mathrm d y \\
&= C \int_0^1 \left(\frac{1}{3}(0.5)^3 + \frac{1}{2}(0.5)^2 y - \frac{1}{3}(0.3)^3 - \frac{1}{2}(0.3)^2 y \right) \mathrm d y \\
&= C \int_0^1 \left(\frac{49}{1500} + \frac{2}{25} y \right) \mathrm d y \\
&= C \left( \frac{49}{1500}y \Big|_0^1 \right) + C \left( \frac{2}{50} y^2 \Big|_0^1 \right) \\
&= \frac{12}{27}\left(\frac{49}{1500} + \frac{2}{50} \right) = \frac{109}{875}
\end{align*}
\item
\begin{align*}
\mathbb{P}(X < Y) &= \int_0^1 \int_0^y (Cx^2 + Cxy) \mathrm d x \mathrm d y \\
&= C \int_0^1 \left( \left( \frac{1}{3}x^3 \Big|_0^y \right) + \left( \frac{1}{2} y x^2 \Big|_{x=0}^y \right) \right) \mathrm d y \\
&= C \int_0^1 \left( \frac{1}{3} (y)^3 + \frac{1}{2} y^3 \right) \mathrm d y \\
&= C \left(\frac{1}{3}\cdot\frac{1}{4} y^4 \Big|_0^1 \right) + C \left(\frac{1}{2}\cdot\frac{1}{4} y^4 \Big|_0^1 \right) \\
&= \frac{12}{7} \frac{1}{12} + \frac{12}{7} \frac{1}{8} = \frac{5}{14}
\end{align*}
\begin{align*}
\mathbb{P}(X < Y) &= \int_0^1 \int_x^1 \left( C x^2 + C xy \right) \mathrm dy \mathrm dx \\
&= C \int_0^1 \left( \left( yx^2 \Big|_{y=x}^1 \right) + \left(\frac{1}{2} xy^2 \Big|_{y=x}^1 \right) \right) \mathrm d y \\
&= C \int_0^1 \left( x^2 + \frac{1}{2} x - x^3 - \frac{1}{2} x^3 \right) \mathrm d x \\
&= C \left( ( \frac{1}{3}x^3 + \frac{1}{4}x^2 - \frac{1}{4}x^4 - \frac{1}{8}x^4) \Big|_0^1 \right) \\
&= C \left( \frac{1}{3} + \frac{1}{4} - \frac{1}{4} - \frac{1}{8} \right) = \frac{12}{7} \cdot \frac{5}{24} = \frac{5}{14} 
\end{align*}
\item
\begin{align*}
\mathbb{P}(X < Y) &= \int_0^1 \int_0^1 \int_0^1 C \left(x^2 + xyz \right) \mathrm d x \mathrm d y \mathrm d z \\
&= C \int_0^1 \int_0^1 \left( (\frac{1}{3}x^3 + \frac{1}{2}x^2 yz) \Big|_0^1 \right) \mathrm d y \mathrm d z 
= C \int_0^1 \int_0^1 \left( \frac{1}{3} + \frac{1}{2} yz \right) \mathrm d y \mathrm d z \\
&= C \int_0^1 \left( (\frac{1}{3}y + \frac{1}{4} y^2 z \Big|_0^1 \right) \mathrm d z
= C \int_0^1 \left( \frac{1}{3} + \frac{1}{4} z \right) \mathrm d z \\
&= C \left( (\frac{1}{3} z + \frac{1}{8} z^2 ) \Big|_0^1 \right)
= C \left( \frac{1}{3} + \frac{1}{8} \right) = \frac{11}{24} C = 1 \\
&\implies C = \frac{24}{11}
\end{align*}
\end{enumerate}


\paragraph{\questionref{q:stats-term}}
\begin{enumerate}[label=\alph*.]
\item A statistic is a function that is computed from data. For example, take a data set $X = \{x_1, x_2, x_3, \dots\}$ where we compute the empirical mean $\bar X = \frac{1}{|X|}\sum_n x_n$.
\item An estimator is a function of data that tries to estimate an unknown quantity. Estimators are statistics. Some statistics are also estimators. For example, if we have some data set from that is sampled from some unknown density $p(x)$, then its mean is unknown, and $\bar X$ is an estimator of it.
\item A consistent estimator finds the correct value of the unknown quantity if the dataset grows to infinity. We will prove that $\bar X$ is a consistent estimate of $\int p(x) x \calcd x$ later on in the course.
\item A sample from a random variable is an outcome of the random experiment it represents. For example, you can have a random variable representing the outcome of a coin toss. A sample from it would be heads or tails. We sampled a random variable independently many times, then the outcomes would occur with the frequency specified by the probability distribution of the random variable. Thinking about sampling outcomes from a random variable is often a helpful conceptual technique to think about randomness.
\end{enumerate}

\subsection{Linear Algebra}

\paragraph{Question \ref{q:dot_product}}
$\x^\top \y = 1 \times 0 + (-2) \times 4 + 5 \times (-3) + (-1) \times 7 = 0 + (-8) + (-15) + (-7) = -30$.

\paragraph{Question \ref{q:matrix_product}}
$\y = (24, -14, -12)^\top$, $|| \x ||_2 = \sqrt{23}$, $|| \y ||_2 = \sqrt{916}$.

Note that by definition the $\ell_2$ norm of a vector is $|| \x ||_2 = \sqrt{\x^\top \x}$.

\paragraph{Question \ref{q:basis}} 1, 2. 

A set of vectors $\{\mathbf{b}_1, ..., \mathbf{b}_K \}$ with $\mathbf{b}_k \in \mathbb{R}^d$ can form a basis of $\mathbb{R}^d$ iff $K = d$ the vectors are linearly independent to each other.

\paragraph{Question \ref{q:span}} 2, 5. 

A point $\x \in \mathbb{R}^d$ is in $span(\{\mathbf{b}_1, ..., \mathbf{b}_K \})$ with $\mathbf{b}_k \in \mathbb{R}^d$ iff we can find $a_1, ..., a_K \in \mathbb{R}$ such that $\x = \sum_{k=1}^K a_k \mathbf{b}_k$.

\paragraph{Question \ref{q:rotation_matrix}} The rotation matrix is 
\begin{equation*}
    \begin{pmatrix}
    \cos{\frac{\pi}{4}} & -\sin{\frac{\pi}{4}} \\
    \sin{\frac{\pi}{4}} & \cos{\frac{\pi}{4}}
    \end{pmatrix}.
\end{equation*}

\paragraph{Question \ref{q:linear_equations}}

a) The matrix $A$ and vector $\mathbf{b}$ are
\begin{equation*}
A = \begin{pmatrix}
1 & 2 & 0 \\
3 & 2 & 4 \\
-2 & 1 & -2
\end{pmatrix}, \quad \mathbf{b} = (2, 5, 1)^\top.
\end{equation*}

b) The inverse of $A$ is 
\begin{equation*}
A^{-1} = \begin{pmatrix}
2/3 & -1/3 & -2/3 \\
1/6 & 1/6 & 1/3 \\
7/12 & 5/12 & 1/3
\end{pmatrix}.
\end{equation*}
Therefore we have $\x = A^{-1} \mathbf{b} = (-1, 3/2, 43/12)^\top$.

c) $\text{rank}(A) = 3$: as $A$ is invertible, it must have full rank.

\paragraph{Question \ref{q:eigen_decomp}}

a) When $A$ is symmetric, then $A = Q \Lambda Q^\top$, and $\x^\top A \x = \x^\top Q \Lambda Q^\top \x = (Q^\top \x)^\top \Lambda (Q^\top \x)$. As $Q$ is an orthonormal matrix, we have $\x \rightarrow Q^\top \x$ a one-to-one mapping. Therefore we have
$$\x^\top A \x = \bm{z}^\top \Lambda \bm{z} = \sum_{i=1}^d \lambda_i z_i^2, \quad \bm{z} = (z_1, ..., z_d)^\top = Q^\top \x.$$
Therefore $\x^\top A \x \geq 0 \Leftrightarrow \sum_{i=1}^d \lambda_i z_i^2 \geq 0$. This is true for any $\x \in \mathbb{R}^{d \times 1}$ if and only if $\lambda_i \geq 0$ for all $i = 1,..., d$.

b) We use the permuation invariance property of matrix trace to show the result:
$$Tr(A) = Tr(Q \Lambda Q^{-1}) = Tr(Q^{-1} Q \Lambda) = Tr(\Lambda) = \sum_{i=1}^d \lambda_i.$$

c) We use the product rule of matrix determinant to show the result:
$$det(A) = det(Q \Lambda Q^{-1}) = det(Q) det(\Lambda) det(Q^{-1}) = det(Q) det(\Lambda) det(Q)^{-1} = det(\Lambda) = \prod_{i=1}^d \lambda_i.$$

d) Let us assume the statement is false, i.e., there exists a solution $\lambda^* \neq \lambda_i, \forall i = 1, ..., d$ for the equation $A \bm{q} = \lambda \bm{q}, \bm{q} \neq 0$. Then we can rewrite the equation as
$$A \bm{q} = \lambda^* \bm{q} \quad \Rightarrow \quad (A - \lambda^* I) \bm{q} = \bm{0} \quad \Rightarrow \quad Q (\Lambda - \lambda^* I) Q^{-1} \bm{q} = 0.$$
By definition, the column vectors of $Q$ forms a basis of $\mathbb{R}^d$. Notice that the diagonal entries of $\Lambda - \lambda^* I$ are non-zero as we assume $\lambda^* \neq \lambda_i$. This indicates a contradiction to the assumption of $\bm{q} \neq 0$:
$$Q (\Lambda - \lambda^* I) Q^{-1} \bm{q} = 0 \quad \Rightarrow \quad Q^{-1}\bm{q} = \bm{0} \quad \Rightarrow \quad \bm{q} = \bm{0}.$$

\section{Answers Lecture 2: Vector Differentiation}
\paragraph{\questionref{q:circle}} Answer discussed in lectures.

\paragraph{\questionref{q:hessian}} TODO
The matrix $\Phi(X)\transpose\Phi(X)$ needs to be invertible. It certainly won't be if $M > N$, since $\rank \Phi(X)\transpose\Phi(X) < M$. However, even if $N>M$, the rank may be deficient if there are linearly dependent rows in $\Phi(X)$. This can happen if you observe repeated input points. A case that is harder to predict and observe is if you observe points that make the feature vectors $\vphi(\vx_n)$ linearly dependent. It would be good to find a good reference in a textbook for when this happens, and what this looks like. However, this final case doesn't happen very often, and usually the inversion is possible if $N \geq M$. And if the inversion is possible, that means that there is no zero eigenvalue in the matrix, and so the matrix is positive definite, and the optimum is proven to be a minimum.

\paragraph{\questionref{q:autodiff}} TODO



% \section{Answers Lecture 4: Probabilistic Modelling Principles}

\paragraph{Question \ref{q:translation}}
\begin{itemize}
    \item[a.] Choose a categorical distribution. Let $T_{\mparam}(\cdot): \mathcal{F} \rightarrow \mathbb{R}^{|\mathcal{E}|}$ maps a French word $f$ to a real-value vector of length $|\mathcal{E}|$. Then define
    $$p(e | T_{\mparam}(f)) = \text{Categorical}(softmax(T_{\mparam}(f))).$$
    \item[b.] The MLE objective is (if we write $e_n$ using one-hot encoding: $e_n = (0, ..., 0, 1, 0, ..., 0)$)
    $$\mparam^* = \arg\max_{\mparam} \frac{1}{N} \sum_{n=1}^N \log p(e_n | T_{\mparam}(f_n)) = \arg\max_{\mparam} \frac{1}{N} \sum_{n=1}^N \log \frac{\exp[T_{\mparam}(f_n)]^\top e_n}{\sum_{i=1}^{|\mathcal{E}|} \exp[T_{\mparam}(f_n)_i] }$$
\end{itemize}

\paragraph{Question \ref{q:clustering_gmm}}
\begin{itemize}
    \item[a.] With i.i.d.~assumption, the MLE objective is:
    $$L(\mparam) = \frac{1}{N} \sum_{n=1}^N \log \sum_{k=1}^K \pi_k \mathcal{N}(x_n; \mu_k, \sigma^2).$$
    \item[b.] The gradient of the MLE objective w.r.t.~$\mu_k$ is
    $$\nabla_{\mu_k} L(\mparam) = \frac{1}{N} \sum_{n=1}^N \frac{\pi_k \mathcal{N}(x_n; \mu_k, \sigma^2)}{\sum_{j=1}^K \pi_j \mathcal{N}(x_n; \mu_j, \sigma^2) } \frac{x_n - \mu_k}{\sigma^2}.$$
    Notice that by Bayes' rule, we have the posterior probability of cluster assignment as
    $$p(k| x_n) = \frac{\pi_k \mathcal{N}(x_n; \mu_k, \sigma^2)}{\sum_{j=1}^K \pi_j \mathcal{N}(x_n; \mu_j, \sigma^2) }.$$
    This means
    $$\nabla_{\mu_k} L(\mparam) = \frac{1}{N} \sum_{n=1}^N p(k | x_n) \frac{x_n - \mu_k}{\sigma^2}.$$
    Setting $\nabla_{k} L(\mparam) = 0$ for all $k$, we have the fixed-point equation as
    $$\mu_k =  \frac{ \sum_{n=1}^N p(k | x_n) x_n}{\sum_{n=1}^N p(k | x_n) }.$$
\end{itemize}


\paragraph{Question \ref{q:linear_regression_projection}}
Using matrix-vector notations we have $\hat{\y} = \bm{\Phi} \mparam^* = \bm{\Phi} (\bm{\Phi}^\top \bm{\Phi})^{-1} \bm{\Phi}^\top \y$. Writing the SVD of $\bm{\Phi} = \bU \Sigma \bV^\top$, it is easy to show that $\bm{\Phi} (\bm{\Phi}^\top \bm{\Phi})^{-1} \bm{\Phi}^\top = \bU \bU^\top$. Notice that $\bU$ contains basis vectors which span to the same subspace spanned by the column vectors of $\bm{\Phi}$.
% \section{Answers Lecture 5: Gradient Descent Convergence}

\paragraph{Question \ref{q:rayleigh_quotient}}

As $\BA$ is symmetric, we can write the eigen-decomposition formula as $\BA = \BQ \Lambda \BQ^{\top}$ with $\BQ$ containing an orthonormal basis of $\mathbf{R}^{d\times 1}$. 
Then using the fact that $\BQ \BQ^\top = \mathbf{I}$, we can define $\z = \BQ^\top \x$ and rewrite the Rayleigh quotient as:
\begin{equation}
R(\BA, \x) = \frac{\x^\top \BQ \Lambda \BQ^\top \x}{\x^\top \BQ \BQ^\top \x} = \frac{\z^\top \Lambda \z}{\z^\top \z}.
\end{equation}
As $\Lambda = \text{diag}(\lambda_1, ..., \lambda_d)$ is a diagonal matrix, we have (writing $\z = (z_1, ..., z_d)^\top$)
\begin{equation}
    \z^\top \Lambda \z = \sum_{i=1}^d \lambda_i z_i^2.
\end{equation}
Therefore the Rayleigh quotient can be written as the following weighted average of the eigenvalues
\begin{equation}
    R(\BA, \x) = \sum_{i=1}^d \frac{z_i^2}{|| \z ||^2_2} \lambda_i, \quad \text{with } \sum_{i=1}^d \frac{z_i^2}{|| \z ||^2_2} = 1.
\end{equation}
In summary, these derivations indicate that the Rayleigh quotient is bounded as
\begin{equation}
\begin{aligned}
    &\lambda_{min}(\BA) \leq R(\BA, \x) \leq \lambda_{max}(\BA) \\
\Rightarrow \quad &\lambda_{min}(\BA) || \x ||_2^2 \leq \x^\top \BA \x \leq \lambda_{max}(\BA) || \x ||_2^2,
\end{aligned}
\end{equation}
where $\lambda_{min}(\BA)$ and $\lambda_{max}(\BA)$ are the smallest and largest eigenvalues of $\BA$, respectively.


\paragraph{Question \ref{q:pre_conditioned_gd}}
For linear regression we have $\nabla_{\mparam} L(\mparam) = \X^\top(\X \mparam_t - \y)$. Therefore in this case the pre-conditioned gradient descent update rule is
$$\mparam_{t+1} =  \mparam_t - \frac{\gamma}{\sigma^2} \BP^{-1} \X^\top(\X \mparam_t - \y).$$
%
Under the formula of Arithmetico-geometric sequence, we have
$$\mparam_{t} = (\mathbf{I} - \frac{\gamma}{\sigma^2} \BP^{-1} \X^\top \X)^t (\mparam_0 - \mparam^*) + \mparam^*, \quad \mparam^* = (\X^\top \X)^{-1} \X^\top \y.
$$
So this means if pre-conditioned gradient descent converges, it will converge to the right answer $\mparam^*$.

Now we need to show that pre-conditioned gradient descent converges with the right choices of $\gamma$ and $\BP$. This requires us to analyse the eigenvalues of the matrix $(\mathbf{I} - \frac{\gamma}{\sigma^2} \BP^{-1} \X^\top \X)^2$, and for a given $\BP$:
$$ \lambda \text{ is an eigenvalue of } \BP^{-1} \X^\top \X \quad \Rightarrow \quad (1 - \frac{\gamma}{\sigma^2})^2 \text{ is an eigenvalue of } (\mathbf{I} - \frac{\gamma}{\sigma^2} \BP^{-1} \X^\top \X)^2.$$
Following the same idea as to prove convergence for gradient descent, we have the learning rates bounds
$$ \gamma_{min} = 2 \sigma^2 / \lambda_{max}(\BP^{-1} \X^\top \X), \quad \gamma_{max} = 2 \sigma^2 / \lambda_{min}(\BP^{-1} \X^\top \X) .$$
Now to set $\gamma_{min} = \gamma_{max}$, we should choose $\BP$ such that $\lambda_{min}(\BP^{-1} \X^\top \X) = \lambda_{max}(\BP^{-1} \X^\top \X)$, and an easy way to do so is to choose $\BP \propto \X^\top \X$.

One can show that for linear regression problems, the Hessian matrix of $L(\mparam)$ is $\nabla^2_{\mparam} L(\mparam) \propto \X^\top \X$. In general for a given loss function $L(\mparam)$ we often set $\BP_t = \nabla^2_{\mparam_t} L(\mparam_t)$ if it can be computed in a fast way.


\paragraph{Question \ref{q:momemtum_gd}}
First note that $\nabla_{\mparam}L(\mparam_t) = \frac{1}{\sigma^2} \X^\top (\X \mparam_t - \y)$
The update equations for both the parameter and the momentum are
\begin{equation}
\begin{aligned}
    \mparam_{t+1} &= \mparam_t - \gamma \nabla_{\mparam} L(\mparam_t) + \alpha \Delta \mparam_t = \mparam_t - \frac{\gamma}{\sigma^2} \X^\top (\X \mparam_t - \y) + \alpha \Delta \mparam_t \\
    \Delta \mparam_{t+1} &= \mparam_{t+1} - \mparam_t =  \alpha \Delta \mparam_t - \frac{\gamma}{\sigma^2} \X^\top (\X \mparam_t - \y).
\end{aligned}
\end{equation}
Now collecting both equations together into a ``joint'' linear equation:
\begin{equation}
\begin{bmatrix}
\mparam_{t+1} \\ \Delta \mparam_{t+1} 
\end{bmatrix} =
\begin{bmatrix}
\mathbf{I} - \frac{\gamma}{\sigma^2} \X^\top \X & \alpha \mathbf{I} \\ - \frac{\gamma}{\sigma^2} \X^\top \X & \alpha \mathbf{I} 
\end{bmatrix} 
\begin{bmatrix}
\mparam_t \\ \Delta \mparam_t 
\end{bmatrix} + 
\begin{bmatrix}
\frac{\gamma}{\sigma^2} \X^\top \y \\ \frac{\gamma}{\sigma^2} \X^\top \y 
\end{bmatrix}.
\end{equation}
Then we can apply the derivation of arithmetico–geometric sequences again, and show that
\begin{equation}
\begin{bmatrix}
\mparam_{t} \\ \Delta \mparam_{t} 
\end{bmatrix} =
\begin{bmatrix}
\mathbf{I} - \frac{\gamma}{\sigma^2} \X^\top \X & \alpha \mathbf{I} \\ - \frac{\gamma}{\sigma^2} \X^\top \X & \alpha \mathbf{I} 
\end{bmatrix}^t 
\begin{bmatrix}
\mparam_0 - \mparam^* \\ \Delta \mparam_0 
\end{bmatrix} + 
\begin{bmatrix}
\mparam^* \\ \bm{0} 
\end{bmatrix},
\end{equation}
with $\mparam^* = (\X^\top \X)^{-1} \X^\top \y$. This equation also says if momentum GD converges, the momentum $\Delta \mparam_t$ will vanish to $\bm{0}$, which is as expected as $\Delta \mparam_{t} = \mparam_{t} - \mparam_{t-1} \rightarrow \bm{0}$.



\printbibliography

\end{document}
